\section{Positive definite matrices}

\textbf{Definition:}\\
A $n \times n$ symmetric matrix $A$ is positive-definite if $\forall$ $x \neq 0, \in \mathbb{R}^n$ $x^TAx > 0$.

\vspace{6pt}


\textbf{Test for positive definiteness}
Each of the following test is a necessary and sufficient condition for a matrix to be positive definite.
\begin{enumerate}
\item $x^TAx > 0$ for all nonzero vectors $x$.\\
Definition.
\item All eigenvalues of $A$ satisfy $\lambda_i > 0$.\\
\textit{Proof:} $x^TAx = \lambda x^Tx = \lambda \|x\|^2$\\
Converse: Every $\forall y > 0,  \in \mathbb{R}^n$, $y = a_1x_1 + \cdots + a_nx_n$\\
$Ay = a_1\lambda_1 + \cdots + a_n\lambda_1$\\
$y^TAy = a_1^2 \lambda_1 + \cdots + a_n^2 \lambda_n > 0$
\item All the upper left submatrices $A_k$ have positive determinants.\\
Clue:
$
x^TAx = 
\begin{bmatrix}
x_k^T & 0	
\end{bmatrix}
\begin{bmatrix}
A_k & * \\
* & * 
\end{bmatrix}
\begin{bmatrix}
x_k\\
0
\end{bmatrix}
= x_k^TA_kx_k
$

\item All the pivots (without row exchanges) satisfy $d_k > 0$.\\
\textcolor{red}{\textit{Proof: Incomplete}}

\end{enumerate}

\vspace{6pt}


\textbf{Theorem: Another test for positive definiteness}\\
The symmetric matrix $A$ is positive definite if and only if:\\
$\exists$ $R$ with independent columns $\ni$ $A = R^TR$.

\vspace{6pt}

\subsubsection{Positive semidefinite Matrices}

The tests for semidefinite matrices will relax to allow zeros.\\
\textbf{Definition:}\\
$\forall$ $x$ $x^TAx \geq 0$


\textbf{Test for positive semi definiteness}

\begin{itemize}
\item $\forall x \neq 0$ $x^TAx \geq 0$.	
\item All eigenvalues satisfy $\lambda_i \geq 0$.
\item No principal submatrices of $A$ have negative determinants.
\item No pivots are negative.
\item $\exists$ $R$, possibly with dependent columns such that $A = R^TR$.

\end{itemize}



\subsection{Singular value decomposition}
Any $m \times n$ matrix $A$ can be decomposed in to:\\

\begin{center}
\framebox{$A = U\Sigma V^T$}	
\end{center}
$\Sigma$: Diagonal matrix, $m\times n$. This diagonal matrix has eignevalues form $A^TA$, not from A.\\
The positive entries $\sigma_1, \sigma_2, \cdots, \sigma_s$, form the first $r$ diagonal elements of $\Sigma$. These elements are called \textbf{singular values}. The remainder of entries in $\Sigma$ is 0.\\
\vspace{6pt}
$U$: Orthogonal matrix, $m \times m$.\\
The columns of $U$ are the eigenvectors of $A^TA$.
\vspace{6pt}
$V$: Orthogonal matrix, $n \times n$.\\
The columns of $V$ are the eigenvectors of $AA^T$.
\vspace{6pt}

\textbf{Remark 1}\\
For positive definite matrices, $\Sigma$ is $\Lambda$ and $U\Sigma V^T$ is identical to $Q\Lambda Q^T$.

\vspace{6pt}

\textbf{Remark 2}\\
$U$ and $V$ give orthonormal bases for all four fundamental subspaces. \\
\begin{tabularx}{\linewidth}{lcrX}
first & $r$ & columns of $U$: & \textbf{column space} of $A$\\
last & $m-r$ & columns of $U$: & \textbf{left nullspace} of $A$\\
first & $r$ & columns of $V$: & \textbf{row space} of $A$\\
last & $n-r$ & columns of $V$: & \textbf{nullspace} of $A$\\
\end{tabularx}


\vspace{6pt}

\textbf{Remark 3}\\
$Av_j = \sigma_ju_j$ or $AV = U\Sigma$



\vspace{6pt}

\textbf{Remark 4}\\
Eigenvectors of $AA^T$ and $A^TA$ goes into columns of $U$ and $V$.\\
$AA^T = (U\Sigma V^T)(V\Sigma^T U^T) = U\Sigma\Sigma^T U^T$, and,\\
$A^TA =  V\Sigma^T\Sigma V^T$.\\




\vfill\null
\pagebreak


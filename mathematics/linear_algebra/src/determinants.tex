\section{Determinant \& Trace}

\subsection{Properties of determinants}
\subsubsection{Three most basic properties of determinants}

\begin{enumerate}
\item det I = 1.
\item Determinant changes sign when two rows are exchanged.
\item Determinant depend linearly on the first row.\\
Eg: 
$$
\begin{vmatrix}
	a + a' & b + b'\\
	c & d	
\end{vmatrix}
=
\begin{vmatrix}
	a & b\\
	c & d	
\end{vmatrix}
+
\begin{vmatrix}
	a' & b'\\
	c & d	
\end{vmatrix}
$$\\

$$
\begin{vmatrix}
	ta & tb\\
	c & d
\end{vmatrix}
=
t
\begin{vmatrix}
	a & b\\
	c & d	
\end{vmatrix}
$$
 
\end{enumerate}

\subsubsection{Derivative properties of determinants}
\begin{enumerate}
	\setcounter{enumi}{3}
	\item $\mid cA \mid = c^n \mid A \mid$.
	\item If two rows of $A$ are equal, then $det A = 0$\\
	Follows from rule 2.
	\item Adding a multiple of one row to another does not alter the value of the matrix.\\
	Follows from rule 3 and rule 4.
	\item If $A$ has a row of $0$s, then $det A = 0$.\\
	Follows form rule 4 and rule 5.
	\item If $A$ is triangular, $det A = \prod d_i$.\\
	Proof: Convert $A$ to an equivalent diagonal determinant by row operation. Now apply rule 3 and finally rule 1.
	\item If $A$ is singular, then $det A = 0$. If $A$ is invertible, then $detA \neq 0$.\\
	Proof:\\
	If $A$ is singular, elimination leads to zero row, $\therefore detA= 0$\\
	If $A$ is invertible, elimination leads to non-zero pivots. $detA = \prod \text{pivots} \neq 0$.
	\item $det(AB) = det(A)\times det(B)$\\
	Proof:\\
	For a diagonal matrix $D$, $det(DB) = det(D)det(B)$\\
	$A$ can be converted to a diagonal matrix $D$ by row operations.\\
	Using the exact same row operation as above $AB$ can be converted to $DB$\\
	\textit{For another proof see book.}
	\item $det(A^T) = det(A)$.\\
	Proof:\\ 
	$PA = LDU$ and $det(PA) = det(LDU)$\\
	$det(A^TP^T) = det(U^TD^TL^T) = det (LDU) = det(PA)$\\
	$det(P^T) = det(P) \therefore det(A^T) = det(A)$.

\end{enumerate}


\subsection{Formulae for determinant}

\subsubsection{Based on LDU decomposition}
If $A$ is invertible $PA = LDU$.\\
$det(A) = \pm det(L)det(D)det(U) = \pm(\text{product of pivots})$\\

\subsubsection{Using linearity of determinants}
Eg: $2\times 2$ case.\\
$$
\begin{vmatrix}
a & b\\
c & d\\
\end{vmatrix}
= 
\begin{vmatrix}
a & 0\\
c & 0
\end{vmatrix}
+
\begin{vmatrix}
a & 0\\
0 & d
\end{vmatrix}
+
\begin{vmatrix}
0 & b\\
c & 0
\end{vmatrix}
+
\begin{vmatrix}
0 & b\\
0 & d
\end{vmatrix}
= ad - bc
$$

\subsubsection{Expansion in cofactors}

\framebox{The determinant of $A$ is any row $i$ times its cofactors}\\
That is, $det(A) = a_{i1}C_{i1} + a_{i2}C_{i2} + \cdots + a_{in}C_{in}$\\
Where , $C_{ij} = (-1)^{i+j}det(M_{ij})$\\
\textbf{Proof:}\\
Split the $n\times n$ matrix to sum of $n\times n$ matrices by linearity property of deteterminants. Each of the constituent matrices can be converted to LDU form.  

\subsubsection{Block matrices}

\begin{itemize}
\item 
$det
\begin{bmatrix}
A & 0 \\
C & D\\ 
\end{bmatrix}
= det(A) det(D) = 
det
\begin{bmatrix}
A & B\\
0 & D\\
\end{bmatrix}
$
\item If $A$ is invertible\\
$
det
\begin{bmatrix}
A & B\\
C & D\\
\end{bmatrix}
 = det(A)det(D - CA^{-1}B)
$
\item If blocks are square matrices of the same size. And $AC = CA$.\\
$det
\begin{bmatrix}
A & B\\
C & D
\end{bmatrix}
 = det(AD - CB)
$

\item If blocks are square matrices of same size and if $A = D$ and $B = C$. Here $A$ and $B$ need not commute.\\
$
det
\begin{bmatrix}
A & B\\
B & A
\end{bmatrix}
 = det(A - B)det(A + B)
$
\end{itemize}

 
\subsection{Application of determinants}


\begin{definition}[Adjugate Matrix]
Transpose of cofactor matrix: $adj(A) = C^T$.
\textcolor{red}{NOTE: $\mid adj(A) \mid = \mid A \mid ^ {n-1}$.}
\end{definition}

\subsubsection{Computation of $A^{-1}$}


$$A^{-1} = \frac{C^T}{det(A)} = \frac{adj(A)}{det(A)}$$
$C$ is cofactor matrix.\\
\textbf{Proof:}\\
$$A^{-1} = \frac{C^T}{det(A)} \implies det(A)I = AC^T$$
Now the diagonal of $AC^T$ will be $det(A)$. To complete the proof we just need to show that the
off-diagonal elements are all 0.\\
Each off-diagonal element is represented as:\\
$a_{i1}C_{j1} + a_{i2}C_{j2} + \cdots + a_{in}C_{jn}$\\
where $i \neq j$\\
This is equal to the determinant of modified $A$ where the $j$'th row 
is replaced by a copy of $i$'th row times $\pm 1$. This determinant $ = 0$\\

\subsubsection{Cramer's rule; Solution to $Ax = b$}

$$x_j = \frac{det(B_j)}{det A}$$
Where, $B_j$ is $A$ with it's j'th column replaced by $b$.\\
\textbf{Proof:} $x = A^{-1}b = \frac{1}{\det(A)}C^Tb$

\subsubsection{Volume of a box}
$det(A)$ represents the volume of a box(higher dimensional parellelogram) whose edges are represented by the row vectors of $A$ (or the column vectors of A).

\subsection{Trace of a Matrix}

\subsubsection*{Basic Properties}

\begin{itemize}
\item \textbf{Linearity:}
    \begin{itemize}
        \item $tr(A + B) = tr(A) + tr(B)$.
        \item $tr(cA) = c \times tr(A)$
    \end{itemize}
\item \textbf{Cyclic properties:}
    \begin{itemize}
        \item $tr(AB) = tr(BA)$.
        \item $tr(ABC) = tr(BCA) = tr(CAB)$.
    \end{itemize}
\item If $B = P^{-1}AP$, then $tr(A) = tr(B)$.
\end{itemize}
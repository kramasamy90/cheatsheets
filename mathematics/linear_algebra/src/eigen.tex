\section{Eigen values and eigen vectors}

\subsection{Introduction}
\textbf{Eigenvalues \& Eigenvectors:}
The scalars $\lambda$ and the vectors $x$ that satisfy $Ax = \lambda x$ are the
eigenvalues and eigenvectors of the matrix $A$.\\
$Ax = \lambda x \implies (A - \lambda I)x = 0 \implies 
\mid A - \lambda I\mid = 0$\\

Setting $\lambda$ as unkown gives rise to a $n$ dimensional polynomial equation 
in $\lambda$ called the (\textbf{Characteristic equation}).\\
Solving this gives eigen values.\\

\textbf{Algebraic multiplicity($M_{\lambda}$):} is the number of times $\lambda$
occurs as a root in the characteristic equation.\\
\textbf{Geometric multiplicity($m_{\lambda}$):} Dimension of the eigenspace 
corresponding to the eigenvalue $\lambda$.\\ 

\begin{mdframed}[backgroundcolor=SlateGray2!40,linecolor=Firebrick4]
\textbf{Key Results:}
\begin{itemize}
\item $\sum \lambda_i = tr(A)$	
\item $\prod \lambda_i = det(A)$
\item For a triangular matrix the Eigen values are along the diagonal.
\item If $Ax = \lambda x$, then $A^kx = \lambda ^k x$.
\item Eigen vectors corresponding to each Eigen value form a vector space.
\item $m_{\lambda} \leq M_{\lambda}$
\end{itemize}
\end{mdframed}

\textcolor{red}{TODO: Proof outlines.}

\subsection{Diagonalization}

If a $n \times n$ matrix $A$ has $n$ independent eigen vectors then, it can be written as follows: 
$$A = S\Lambda S^{-1} $$ 
where, each column of S is one of the eigenvectors of $A$ ($e_i$)\\

\begin{mdframed}[backgroundcolor=blue!10,linecolor=blue]
\begin{center}
\textbf{Caley-Hamilton's theorem.}	
\end{center}
\begin{theorem}[]
$A$ satisfies it's own characteristic equation.
\begin{proof}
\hspace{5pt}\\
Substitute $A = S\Lambda S^{-1}$ in $(A -\lambda_1 I)(A - \lambda_2 I)\cdots(A - \lambda_n I)$.
\end{proof}
\end{theorem}
\end{mdframed}
\textcolor{red}{TODO: Proof for case where $A$ is not diagonalizable.}



\subsection{Difference equations and powers $A^k$}
\textbf{Difference equations}
$$u_{k+1} = Au_k$$
In some sense difference equations are analogous to differential equations.

\subsubsection{Examples}

\textbf{Fibonacci numbers}\\

$F_{k+2} = F_{k+1} + F_k$\\

Let, 
$
u_k = 
\begin{bmatrix}
F_{k+1}\\
F_k
\end{bmatrix}
$
then, difference representation of Fibonacci sequence would be:\\
$
u_{k+1} = 
\begin{bmatrix}
1 & 1 \\
1 & 0
\end{bmatrix}
\begin{bmatrix}
F_{k+1}\\
F_k	
\end{bmatrix}
= Au_k
$\\
The solution to this difference equation, $u_{k+1} = Au_k$ is \\
\begin{center}
\framebox{$u_k = A^ku_0$}
\end{center}
If $A$ can be diagonalized then $u_k = S\Lambda^kS^{-1}u_0$. \\

\vspace{6pt}

\textbf{Markov matrices}\\
Markov matrix is similar to the above examples along with the following two properties:\\
\begin{enumerate}
\item Each column of Markov matrix adds to 1.
\item The numbers outside and inside can never become negative.
\end{enumerate}
$u_{k+1} = Au_k$  and the solution is $u_k = A^ku_0$\\
If A can be diagonalized, then $u_k = S\Lambda^kS^{-1}u_0$.
\textit{Steady state}($u_{\infty}$): $Au_{\infty} = u_{\infty}$.

\textit{Properties of a Markov matrix A:}\\
\begin{enumerate}
\item $\lambda_1= 1$ is an eigenvalue of $A$.
\item Its eigenvector $x_1$ is nonnegative and $Ax_1 = x_1$.
\item The other eigenvalues satisfy $|\lambda_1| \leq 1$.
\item If $A$ or any power of a $A$ has all positive entries, these other $|\lambda_i|$ are below 1. 
\item The solution $A^ku_0$ approaches a multiple of $x_1$ which is the steady state $u_{\infty}$.
\end{enumerate}




\textbf{Stability of $u_{k+1} = Au_k$}\\

The difference equation $u_{k+1} = Au_k$ is:\\
\begin{itemize}
\item \textit{stable} if all eigenvalues satisfy $|\lambda_i| < 1$.
\item \textit{neutrally stable} if some $|\lambda_i| = 1$ and all other $|\lambda_i| < 1$.
\item \textit{unstable} if at least one $|\lambda_i| > 1$.
\end{itemize}


\subsection{Differential equations and $e^{At}$}
\subsubsection{System of differential equations}
Matrices are useful to solve a system of differential equations.\\
Eg:
\begin{align*}
\frac{du_1}{dt} &= a_{11}u_1 + a_{12}u_2 + \cdots + a_{1n}u_n\\
\frac{du_2}{dt} &= a_{21}u_1 + a_{22}u_2 + \cdots + a_{2n}u_n\\
&\vdotswithin{=}\\
\frac{du_n}{dt} &= a_{n1}u_1 + a_{n2}u_2 + \cdots + a_{nn}u_n\\
\end{align*}

This system in matrix notation is $\frac{du}{dt} = Au$, where\\
$A = 
\begin{bmatrix}
a_{11} & \cdots & a_{1n}\\
\vdots & & \vdots\\
a_{n1} & \cdots & a_{nn}	
\end{bmatrix}
$

\textit{Solution:}\\
$$u(t) = e^{At}u(0)$$
\textcolor{red}{Is the above solution valid if $A$ is not diagonalizable.}\\
\textit{Proof:}
If $A$ is diagonalizable then let $v = S^{-1}u$, where $S$ is the eigenvector matrix of A. Using eigenvector matrix the system of linear equations is decoupled.\\ 
\begin{align*}
\therefore S\frac{dv}{dt} &= ASv\\
\implies \frac{dv}{dt} &= S^{-1}ASv = \Lambda v\\
v(t) &= e^{\Lambda t}v(0)\\
u(t) &= Se^{\Lambda t}S^{-1}u(0) = e^{At}u(0)\\
\end{align*}
where $e^{At}$ is the exponential of the matrix $A$.\\




\subsubsection{Matrix exponential}

\textbf{Matrix exponential} is defined as follows:\\
$$e^{At} = I + At + \frac{(At)^{2}}{2} + \cdots + \frac{(At)^n}{dt} + \cdots$$\\

This series \textbf{always converges} and has the following properties:\\
\begin{itemize}
\item $(e^{As})(e^{At}) = e^{A(s+t)}$\\
\item $(e^{At})(e^{-At}) = I$\\
\item $\frac{d}{dt}(e^{At}) = Ae^{At}$
\item If $A$ can be diagonalized, $A = S\Lambda S^{-1}$, then $du/dt = Au$ has the solution $u(t) = e^{At}u(0)$.\\
\item $e^{At}$ is never singular.\\
\textit{Proof 1:} If $\lambda$ is an eigenvalue of $A$, then $e^{\lambda t}$ is the corresponding eigenvalue of $e^{At}$, which is never 0.\\
\textit{Proof 2:} $det(e^{At}) = e^{\lambda_1 t}e^{\lambda_2 t}\cdots e^{\lambda_n t} = e^{trace(At)} \neq 0$.
\end{itemize}


\subsubsection{Stability of differential equations}
Behaviour of $u(t)$ as $t \to \infty$:
The differential equation $du/dt = Au$ is:\\
\begin{itemize}
\item \textbf{stable:} and $e^{At} \to 0$ whenever, all $Re(\lambda_i) < 0$.	
\item \textbf{neutrally stable:} when all $Re(\lambda_i) < 0$ and $Re(\lambda_1) = 0$.
\item \textbf{unstable:} and $e^{At}$ is unbounded if any eigenvalue has $Re(\lambda_i) > 0$.
\end{itemize}
These results are true even if $A$ is not diagonalizable.


\subsection{Complex matrices}

\textbf{Length of a vector:} $\|x\| = |x_1| + |x_2| + \cdots + |x_n|$\\

\subsubsection{Hermitian (and symmetric) matrices}

\textbf{Conjugate transpose:} $\overline{A}^T = A^H$ (Read as A Hermitian) has entries $(A^H)_{ij} = \overline {A_{ji}}$

\textbf{Hermitian matrices:} $A$ is Hermitian iff $A^H = A$.\\

\textbf{Theorem:} If $A = A^H$ then $\forall$ $x$, $x^HAx$ is real.\\
\textit{Proof:} $(x^HAx)^H = x^HA^Hx = x^HAx$.\\

\textbf{Theorem:} If $A = A^H$ then all the eignevalues are real.\\
\textit{Proof:} $x^HAx = \lambda x^Hx$. LHS is real and $x^Hx$ is real, therefore $\lambda$ is real.\\

\textbf{Theorem:} Eigenvectors of Hermitian matrices that come from two different eigenvalues are orthogonal.\\
\textit{Proof:} \\
Let, $Ax = \lambda x$ and $Ay = \mu y$\\
$\lambda x^Hy = (Ax)^Hy = x^HAy = x^H \mu y \implies x^Hy = 0, \because \lambda \neq \mu$\\

\textbf{Orthonormal matrices:} $QQ^T = I$.

\textbf{Theorem (Spectral theorem):} A real symmetric matrix can be factored into \framebox{$A = Q\Lambda Q^T$}. Its orthonormal eigenvectors are in the columns of $Q$.

\subsubsection{Unitary matrices}

\framebox{$U^HU = I$}\\
Multiplication by $U$ has no effect on inner products, angles and lengths.\\
\textbf{Property 1:}\\
\begin{itemize}
\item $(Ux)^T(Uy) = x^TU^TUy = x^Ty$\\
\item $\|Ux\|^2 = x^HU^HUx = x^Hx = \|x\|^2$
\end{itemize}
\textbf{Property 2:}\\
For every eigenvalue($\lambda$) of $U$, $|\lambda| = 1$\\

\textbf{Property 3:}\\
Eigen vectors corresponding different eigenvalues are orthogonal.\\
Let, $Ux = \lambda x$ and $Uy = \mu y$.\\
$x^Hy = (Ux)^H(Uy) = (\lambda x)^H(\mu y) = \overline{\lambda}\mu x^Hy$\\
By property-2 $\overline{\lambda}\lambda = 1, \therefore \overline{\lambda}\mu \neq 1$\\
$\implies x^Hy = 0$.

\vspace{6pt}

\framebox{If $A$ is Hermitian then $K =  iA$ is skew-Hermitian.}\\

\textbf{Theorem:} Eigenvalues of $K$ are purely imaginary.\\

\subsubsection{Fourier matrix}
\textcolor{red}{THE FOLLOWING IS INCOMPLETE}\\


$$F_n =
\begin{bmatrix}
1 & 1 & 1 & \cdots & 1\\
1 & \omega & \omega^2 & \cdots & \omega^{n-1}\\
1 & \omega^2 & \omega^4 & \cdots & \omega^{2(n-1)}\\
\vdots & \vdots & \vdots & \ddots & \vdots \\
1 & \omega^{n-1} & \omega^{2(n-1)} & \cdots & \omega^{(n-1)^2}\\
\end{bmatrix}
$$


\subsection{Similarity transformation}
\textbf{Definition:} $A$ and $B$ are similar if $\exists M \ni B= M^{-1}AM$.\\
Going from one to another is known as \textbf{similarity transformations}.\\

\vspace{6pt}
In the case where $M = S$, $M^{-1}AM$ becomes the diagonal matrix $\Lambda$. This is the best case scenario. But other $M$'
s are also useful. Usually $M$ is chosen such that, $M^{-1}AM$ is easier to work with than $A$.
\vspace{6pt}

\textbf{Theorem:} Suppose $B = M^{-1}AM$, then $A$ and $B$ have the same eigenvalues.\\
\textit{Proof:} $Ax = \lambda x \implies B(M^{-1}x) = \lambda (M^{-1}x)$\\
$M^{-1}x$ is the eigenvector of $B$ corresponding to $\lambda$.
\vspace{6pt}

\subsubsection{Change of Basis = Similarity transformation}

Every linear transformation is represented by a matrix. Similar matrices represent the same transformation $T$ with respect to different basis.\\
\begin{tabularx}{\linewidth}{lllX}
$[T]_{\text{V to V}}$ 	& $= [I]_{\text{v to V}}$ & $[T]{\text{v to v}}$ & $[I]_{\text{V to v}}$\\
$B$ & $=M^{-1}$ & $A$ & $M$
\end{tabularx}

\subsubsection{Triangular forms with unitary M}

\textbf{Shcur's lemma:}\\
There is a unitary matrix $M = U$ such that $U^{-1}AU = T$ is triangular. The eigenvalues of A appear along the diagonal of this similar matrix $T$.


\subsubsection{Diagonalizing Hermitian matrices}

\textbf{Theorem:}\\
If $A = A^H$ This triangular form $T$, is a diagonal matrix.\\
\textit{Proof:} $(U^{-1}AU)^H = U^HA^H(U^{-1})^H = U^{-1}AU \implies T = T^H$.

\vspace{6pt}

\textbf{Spectral theorem:}\\
Every real symmetric matrix can be diagonalized by an orthogonal matrix $Q$. Every Hermitian matrix can be diagonalized by a unitary matrix $U$:\\
\begin{center}
\begin{tabular}{rccc}
(real) & $Q^{-1}AQ = \Lambda$ & or & $A = Q\Lambda Q^T$\\
(complex) & $U^{-1}AU = \Lambda$ & or & $A = U\Lambda U^H$
\end{tabular}
\end{center}

\vspace{6pt}

\textbf{Normal matrices} \\
The matrix $N$ is normal if it commutes with $N^H$: $NN^H = N^HN$.


\vspace{6pt}

\textbf{Theorem:}\\
A triangular matrix $T$ that is normal must be diagonal\\
\textit{Proof:}\\
Tip: Use induction and multiplication by blocks.\\
For details see: \url{https://math.stackexchange.com/a/2538528/633346}



\vspace{6pt}

\textbf{Theorem:}\\
$T = U^{-1}AU$ is diagonal if and only if $A$ is a normal matrix.\\
\textit{Proof:}\\
If $A$ is normal, \\
$TT^H = U^{-1}AUU^HA^HU = U^{-1}AA^HU = U^{-1}A^HAU = U^HA^HUU^{-1}AU = T^HT \implies T$ is diagonal.

\subsubsection{Jordan form}

\textbf{Theorem:} If $A$ has $s$ independednt eigenvectors, it is simillar to a matrix with $s$ blocks.\\
$$
\text{\textbf{Jordan form }}
J = M^{-1}AM =
\begin{bmatrix}
J_1 & & \\
& \ddots & \\
& & J_s\\
\end{bmatrix}
$$

Each Jordan block is a triangular matrix that has only a single eigenvalue ($\lambda_i$) along the diagonal corresponding to only one eigenvector. For each missing eigenvector there will be a $1$ just above the diagonal.


$$
\text{Jordan block }
J_i = 
\begin{bmatrix}
\lambda_i & 1 & & \\
& \lambda_i &  \cdot & \\
& & \cdot & 1\\
& & & \lambda_i
\end{bmatrix}
$$

\section{Matrices and Gaussian Eliminations}

\subsection{Geometry of linear equations}

Matrix representation: $Ax = b$ \\%

\begin{itemize}
\item \textit{Row picture}: Lines in a ($\mathbb{R}^2$) or planes ($\mathbb{R}^3$).
\item \textit{Column picture}: $b$ is linear combination of the columns of A. $\therefore$ $b$ is in the column space of A. 
\end{itemize}

\textbf{Singular case:} No solution or infinite solution.\\

\subsection{Gaussian Elimination}
\begin{itemize}
	\item Apply row operations and row exchange to convert the system of linear equations to triangular form.
	\item Use back substitution to solve the system.
\end{itemize}

\textcolor{red}{Row operations  preserve the nullspace and rowspace of $A$,\\ but this normally alters the eigenvalues.}

\subsubsection{Gaussian elimination in matrix form}

\begin{itemize}
	\item Each row operation on $A$ can be represented as a matrix multiplication on A. If a triangular form $U$ is obtained by three row operations on A, each of them can be represented by matrices, say $G, F$ and $E$.\\
	Then, $G \times F \times E \times A = U$.
	\item The product of $G, F$ and $E$ would be a lower triangular matrix (L). The matrix $A$ can be written as $PA = LU$. $L$ can be found by keeping track of the row operations. $P$ is a permutation matrix.
	\item $LU$ can also be written as $LDU$ by splitting $U$ to $DU$. $D$ is a diagonal matrix containing the pivot elements.
\end{itemize}

\subsection{Matrix multiplication}

\subsubsection{5 ways to multiply matrices}

Given, $A \times B = C$\\

\begin{itemize}
\item Standard.
\item Each column of $C$ is a linear combination of columns of A.\\
	$\begin{bmatrix}
		Ab_1 & \cdots & A b_n 
	\end{bmatrix}$
	, where $b_i$s are columns of $B$. 
\item Each row of $C$ is a linear combination of rows of B.\\
	$\begin{bmatrix}
		a_1B\\
		\vdots\\
		a_nB
	\end{bmatrix}$
	, where $a_i$s are rows of $A$.
\item $\sum_{i=1}^{n} [ca_i] \times [rb_i]$\\
Where, $ca_i$ and $rb_i$ are $i$-th column and row of $A$ and $B$, respectively.
\item Multiplying Blocks of matrix. Eg:\\
	$
	\begin{bmatrix}
		A_1 & A_2\\
		A_3 & A_4
	\end{bmatrix}	
	\times
	\begin{bmatrix}
		B_1 & B_2\\
		B_3 & B_4
	\end{bmatrix}	
	=
	\begin{bmatrix}
		C_1 & C_2\\
		C_3 & C_4
	\end{bmatrix}	
	$\\
$A_i, B_i$ and $C_1$ are blocks of the matrices $A, B$ and $C$. The blocks are treated as a single entity and multiplied like a normal matrix. Eg:
$C_1 = A_1B_1 + A_2B_3$\\
\textcolor{red}{Last approach would be useful for proving matrix theorems by inductions.}
\end{itemize}

\subsubsection{Properties}

\begin{itemize}
	\item Associative: $(AB)C = A(BC)$.
	\item Distributive: $A(B+C) = AB + AC$; $(B+C)D = BD + CD$.
	\item \textbf{NOT} commutative: Usually $EF \neq FE$.
	\item $AB = 0 \centernot\implies A=0 \text( or ) B=0$.\\
	\item $AC = AD \centernot\implies C = D$\\
	\item Let $A, B, C$ be $n\times n$ matrices. Then:\\
	\item If $rank(A) = n$ and $AB = AC$, then $B=C$.\\
	\item If $rank(A) = n$, then $AB=0 \implies B = 0$. If $AB = 0$ but $A \neq 0$ and $B \neq 0$, then $rank(A) < n$ and $rank(B) <n$.
\end{itemize}





\subsection{Inverse and Transpose of a Matrix}
If the system of equations is non-singular, then it could be solved by using inverse of $A$.
\begin{itemize}
\item $A^{-1}$ is inverse of $A$ $\Leftrightarrow$ $A^{-1}A = A^{-1}A = I$.
\item If $A^{-1}$ exist, then A is invertible or non-singular matrix.
\end{itemize}

\subsubsection{Properties}

\begin{itemize}
\item A square matrix is singular or not invertible if:
	\vspace{-6pt}
	\begin{itemize}
	\item Its determinant is $0$.
	\item Dimension(Column Space) $<$ Matrix dimension.\\
	\textit{Proof (by contradiction):} $\exists x \neq 0, \ni Ax = 0$.\\ 
	Assuming $A^{-1}$ exists $\Rightarrow$ $A^{-1}Ax = A^{-1}0 \Rightarrow x = 0$. 
	\end{itemize}

\item If $A$ is invertible, then the one and only solution to $Ax = b$ is $x = A^{-1}b$
\item $(AB)^{-1} = B^{-1}A^{-1}$
\item $(AB)^T = B^TA^T$
\item $(A^{-1})^T = (A^T)^{-1}$
\item For any rectangular matrix $R$, $RR^T$ and $R^TR$ is symmetric.
\item A symmetric matrix could be decomposed to $LDL^T$.
\end{itemize}

\subsubsection{Finding inverse by Gauss-Jordan elimination}


Start with augmented matrix. Achieve upper triangular matrix on the left part and then identity matrix by row operations:\\
\vspace{-4pt}
\begin{center}
\framebox{$[A|I] \longrightarrow [U|X] \longrightarrow [I|A^{-1}]$
}
\end{center}


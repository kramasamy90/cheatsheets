\section{Preliminaries}

\subsection{Torch Tensor}

\subsubsection*{Creating a tensor.}
\texttt{torch.<method>(<options>)}\\
\begin{tabularx}{\linewidth}{lX}
    \hline
    \texttt{arange(l, r, s)}        & $(l, l+s, \cdots, l+ks) \ni l+ks < r$ and $k$ is the largest such integer.\\
                                    & Default: $l = 0$, $s = 1$.\\ 
    \texttt{linspace(l, r, n)}      & $(l, l+s, \cdots, r(= l + n\times s)) \& s = (r-l)/n$.\\
    \texttt{ones(<shape>)}          & \\ 
    \texttt{zeros(<shape>)}         & \\ 
    \texttt{full(<shape>, k)}       & Constant tensor filled with each element $= k$.\\ 
    \texttt{randn}                  & \\
    \hline
    \texttt{tensor}                 & From NDArray.\\
                                    & Eg: \texttt{torch.tensor([1, 2, 3])}.\\
    \texttt{from\_numpy}            & \\
    \hline
\end{tabularx}

Other related methods: \texttt{zeros\_like}, \texttt{ones\_like}, \texttt{empty\_like}.\\
Eg: \texttt{y = ones\_like(x)}.\\

\subsubsection{Tensor Properties \& Operation}

\begin{tabularx}{\linewidth}{lX}
    \hline
    \texttt{numel()}    & Number of elements. Eg, a $2 \times 2$ tensor has 4 elements.\\
    \texttt{dtype}      & \\
    \texttt{shape}      & \\
    \texttt{reshape()}  & Eg: \texttt{torch.arange(8).reshape(4, 4)}\\
                        & Eg: \texttt{torch.arange(8).reshape(4, -1)}\\
                        & \textcolor{blue}{Use $-1$ to automatically infer one of the dimensions.}\\
    \texttt{numpy()}    & \\
    \texttt{item()}     & Can be applied only to a tensor with single element. Returns the element.\\
                        & Single element tensors can also be coverted as follows:\\
                        & \texttt{int(x), float(x)}, etc.\\
    \hline
    \texttt{cat}        & Concatenate along a dimension. Specify \texttt{dim}.\\
    \texttt{sum}        & \\
\end{tabularx}

\textbf{Operations}\\



\subsubsection*{Uncategorized}

\begin{tabularx}{\linewidth}{lX}
    
\end{tabularx}
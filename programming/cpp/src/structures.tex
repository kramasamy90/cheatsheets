\section{Structures, unions and typedefs}

\subsection{Structure}

\begin{itemize}
\item \textbf{Definition:}\\
\texttt{struct point}\\
\texttt{\{}\\
\qquad \texttt{int x;}\\
\qquad \texttt{int y;}\\
\texttt{\};}\\

NOTE-1: Structure tag is optional??\\
NOTE-2: In C  a function cannot be a member of a structure.\\

\item \textbf{Declaration examples:}\\
\begin{itemize}
\item \texttt{struct \{$\cdots$\} a,b,c;}\\
\item \texttt{struct point \{$\cdots$\} a,b,c;}\\
\item \texttt{struct point a,b,c}\\
\item \texttt{struct point *p;}\\
\end{itemize}

\item \textbf{Accessing members}\\
\texttt{a.x = 5;}\\
\texttt{printf("\%d\textbackslash n",a.x);}\\

\item \textbf{Accessing members with pointer}\\
\texttt{struct point *p = \&a;}\\
The following are equivalent:\\
\begin{itemize}
\item \texttt{p -> x;}\\
\item \texttt{(*p).x;}\\
\item \texttt{a.x;}
\end{itemize}

\item \textbf{Arrays of structure:}\\
\texttt{struct points pts[100];}\\
\texttt{struct \{$\cdots$\} pts[] = \{\{$\cdots$\},\{$\cdots$\},$\cdots$,\{$\cdots$\}\};}\\
NOTE: In above assignment, in the RHS, elements within the braces could be of different types matching the members of the structure.\\
In case of simple members, each members need not be enclosed within braces.\\
 
\end{itemize}

\subsection{typedef}
end{itemize}
\begin{itemize}
	\item
		\texttt{typedef int Length;}\\
		\texttt{Length len, maxlen;}\\
		\texttt{Length *length[];}\\
		\texttt{typedef struct tnode *Treeptr;}\\
		\texttt{typedef struct tree Treenode;}\\
	\item The above examples could also be implemented by \texttt{\#define}\\
		The following can only be implemented by \texttt{typedef}:\\
		\texttt{typedef int (*PFI) (char*, char*);}
\end{itemize}

\subsection{Union}
A union is a variable that may hold objects of different types and sizes.\\
The syntax is based on structures:\\
\texttt{union u\_tag \{}\\
\texttt{\qquad int ival;}\\
\texttt{\qquad float fval;}\\
\texttt{\qquad char *sval;}\\
\texttt{\} u;}\\

For example, integer value of u can be accessed as:\\
\texttt{u.ival}

\subsection{Bit fields}

Eg:\\
\texttt{
	struct \{\\
		\qquad unsigned int is\_keyword : 1;\\
		\qquad unsigned int is\_extern : 1;\\
		\qquad unsigned int is\_static: 1;\\
	\} flags;
}\\

This defines \texttt{flags} that contains three 1-bit fields.\\
The number following the colon represents the field width.\\

\textbf{Namespace}
\texttt{using namespace std;}\\
	With this statement standard library functions can be used without the prefix \texttt{std::}\\

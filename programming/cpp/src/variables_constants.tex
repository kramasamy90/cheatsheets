\section{Variables and Constants}

\subsection{Variable types}
\begin{tabularx}{\linewidth}{lllX}
Type & Location & memory location & Scope\\
\hline
\texttt{Global} & Outside functions & data/bss & Global\\
\texttt{Static} & Outside functions & data/bss & Source file\\
\texttt{Static} & Inside a function & data/bss & Local\\
\texttt{Local} & Inside a function & stack & Local\\
\texttt{Register$^\#$} & Inside a function$^*$ & register & Local\\
\hline
\end{tabularx}
$^*$ causes error if declared in global space.\\
$^\#$ This is not strict. The compiler might choose not to keep the variable in register.

\subsection{Variable declaration and initialization}
\begin{itemize}
\item Eg: \texttt{int num;}
Local or global depending on context.
\item Eg: \texttt{static int num;}
Static variable.
\item Eg: \texttt{const int num;}
Causes error if \texttt{num} is modified.
\end{itemize}
\subsection{Data types}
\textbf{Major variable types}
\begin{tabularx}{\linewidth}{lX}
\texttt{char} & 1 byte.\\
\texttt{short int} & 2 bytes.\\
\texttt{int} & 4 bytes.\\
\texttt{long int} & 4 to 8 bytes.\\
\texttt{long long int} & 8 bytes.\\
\texttt{\_\_int128\_t} & 16 bytes, not part of standard definition, but is supported by g++.\\ 
\texttt{float} & 4 bytes.\\
\texttt{double} & 8 bytes.\\
\texttt{long double} & 80 bit floating point supported by g++.\\
\texttt{size\_t} & unsigned type.\\
\end{tabularx}
Their actual size might vary depending on the system.\\

\textbf{Modifiers for variable types}
\begin{tabularx}{\linewidth}{lX}
\texttt{unsigned} & With int and char.\\
\texttt{signed} & With int and char.\\
\end{tabularx}


\subsection{Constants}

\begin{tabularx}{\linewidth}{lX}
\texttt{1234} & \texttt{int}\\
\texttt{1234567890L} & \texttt{long int}. \textcolor{red}{When should I use L?}\\
\texttt{010} & Octal $\equiv$ 8.\\
\texttt{0x10} & Hexadecimal $\equiv$ 16.\\
\texttt{0b10} & Binary $\equiv$ 2. Binary format is not part of standard C, but is supported by gcc.\\
'x' & Character constant, has an integer value.\\
'\textbackslash ooo' & Specify ASCII code of the character in octal.\\
'\textbackslash xhh' & Specify ASCII code of the character in hexadecimal.\\
\end{tabularx}

\textbf{Escape sequences:}\\
 \texttt{\textbackslash a, \textbackslash b, \textbackslash f, \textbackslash n, \textbackslash r, \textbackslash t, \textbackslash v, \textbackslash \textbackslash, \textbackslash ?, \textbackslash ', \textbackslash "}.

\textbf{Enumeration constants}: List of constant integers.\\

\begin{itemize}
\item Eg: \texttt{enum ans \{NO, YES\};}\\
By defaults integers are assigned from 0.
\item Eg: \texttt{enum days \{MON=1,TUE,WED,THU,FRI,SAT,SUN\};}\\
Here, \texttt{MON} is assigned 1, and by default, \texttt{TUE} is 2. 
\item Eg: \texttt{enum months \{JAN =1, FEB, APR=4,MAY\};}\\
Here, \texttt{MAY} is 5.
\end{itemize}

\section{Compile and Debug}

\subsection{G++}
Usage:\\
\texttt{g++ \textit{option} hello.cpp -o hello}\\
\texttt{g++ hello.cpp -o hello}\\
\texttt{g++ -O3 hello.cpp -o hello}\\
Options:\\
\begin{tabularx}{\linewidth}{lX}
\texttt{-o} & Output file name.\\
\texttt{-I} & Include additional directories in the search path. Eg:\\
& \texttt{g++ -I /home/user/libs foo.cpp}\\
\texttt{-O\textit{<level>}} & Optimization level, starts from 0. Eg: \texttt{-O3}\\
\texttt{-g} & Creates breakpoints for debugging using GDB.\\
\texttt{-lm} & Missing link. Used to link certain libraries like \texttt{math.h}.\\
\texttt{-E} & Output preprocessed but uncompiled code.\\
\texttt{-S} & Compile but do not assemble.\\
\texttt{-c} & Compile and assembly, but do not link.\\
\end{tabularx}

\subsection{GDB}
To run: \texttt{gdb hello}\\
\begin{tabularx}{\linewidth}{lX}
\texttt{break} & Set break point. Eg: \texttt{break \textit{function}} or \texttt{break \textit{line number}}.\\
\texttt{run} & Run the program. The program stops at every break point.\\
\texttt{next} & Run until next breakpoint.\\
\texttt{print} & Eg: \texttt{print i} or \texttt{print \&i}\\
\texttt{sizeof} & Eg: \texttt{print sizeof(i)}\\
\texttt{\&i} & Address of i.\\
\texttt{*j} & Content of memory location j.\\
\texttt{ptype} & get type of a variable. Eg: \texttt{ptype(i)}\\
\texttt{set var} & Reassign variable. Eg: \texttt{set var i = 1}\\
\texttt{disassemble} & Use after break and run. Gives assembly code.\\
\end{tabularx}

\textbf{Accessing memory location using x}\\
Usage: \texttt{x/nfs}. \texttt{nfs} describes the format.\\
n - Number of units to display.\\
f - Number format.\\
s - Size of each unit.\\
\begin{tabularx}{\linewidth}{lX}
\hline
\texttt{x} & Hex.\\
\texttt{o} & Octal.\\
\texttt{t} & Binary.\\
\texttt{d} & Decimal.\\
\texttt{u} & Unsigned decimal.\\
\texttt{i} & Instruction.\\
\texttt{c} & Character.\\
\texttt{s} & String.\\
\hline
\texttt{b} & byte.\\
\texttt{h} & Halfword.\\
\texttt{w} & Word.\\
\texttt{g} & Giant.\\
\hline
\end{tabularx}


\section{Pointers and arrays}

\subsection{Pointers}

\begin{tabularx}{\linewidth}{l|X}
\hline
Declaration & \texttt{\textit{type} *\textit{ptr};}\\
& Eg: \texttt{int *p;}\\
\hline
Initialization & \texttt{\textit{type *ptr = val};}\\
& Eg: \texttt{int *p = 0;}\\
& Eg: \texttt{int *p = NULL;}// Null pointer.\\
& Eg: \texttt{int *p = (int *) 100;}\\
& NOTE: Casting is required for integer other than zero. Also, the above memory location may not be available.\\
\hline
\texttt{\&} & Gives address.\\
& Eg: \texttt{ptr = \&x;}\\
\hline
\texttt{*} & Dereferencing operator.\\
& Eg, \texttt{*ptr} refers to x.\\
\hline
\end{tabularx}

\subsection{Arrays}

\begin{itemize}
\item \textbf{Character arrays}\\
\texttt{char s[100];}\\
\texttt{s = "abc";} //ERROR.\\
\texttt{*s = "abc";} // OK.\\
\texttt{char s[] = \{'a','b','c'\};}\\
\texttt{char s[] = "abc";}\\
\item \textbf{Integer, float arrays}\\
\texttt{int num[100]}\\
\texttt{num = \{1,2,3\}} // ERROR\\
\texttt{*num = \{1,2,3\}} // ERROR \\
\texttt{int num[] = \{1,2,3\}}\\
\texttt{float num[] = \{1.0,2.0,3.0\}}\\
\end{itemize}

\subsection{Character pointers}

\textbf{Examples:}\\

\texttt{char *pmessage;}\\
\texttt{pmessage = "Hello, World!\textbackslash n";} \\
\texttt{printf("\%s",pmessage); \# Prints the string}\\
\texttt{*pmessage} refer to '\texttt{H}'\\

\subsection{Pointer to pointers}

\textbf{Examples:}\\

\begin{tabularx}{\linewidth}{l|X}
\texttt{int **p} & \texttt{p} is a pointer to a pointer to \texttt{int}\\
& \texttt{*p} is poiner to a pointer to \texttt{int}\\
\texttt{char *line[MAXLEN]} & \texttt{line} is a pointer to a character array.\\

\end{tabularx}

\subsection{Multi-dimensional arrays}

I think, multi-dimensional arrays can be thought of as pointers to pointers etc.\\

Example:\\
Given: \texttt{int a[2][3] = \{\{1,2,3\},\{4,5,6\}\} } \\
The following are equivalent:\\

\begin{tabular}{l|l}
\hline
\texttt{a[0][0]} & \texttt{**a}\\
\hline
\texttt{a[1][1]} & \texttt{*(*(a + 1) + 1)}\\ 
\hline
\end{tabular}

\hspace{10pt}\\

Here \texttt{a} is a pointer to an array of pointers. 

\subsection{Arrays and pointers}
The following two are equivalent:\\

\begin{tabular}{l|l}
\hline
\texttt{pa = \&a[0];} & \texttt{pa = a;}\\
\texttt{a[i]} & \texttt{*(a + i)}\\
\texttt{p[i]} & \texttt{*(p + i)}\\
\texttt{f(int arr[])} & \texttt{f(int *arr)}\\
\hline
\end{tabular}

\begin{tabular}{l|l}
\hline
Legal & Illegal\\
\hline
\texttt{pa++} & \texttt{a++}\\
\texttt{pa[-1]} & \texttt{a[-1]}\\
\hline
\end{tabular}

\subsection{Pointer to functions}



Eg: \\
\texttt{int sum\_f(int size, int arr[], int (*foo)(int ))} // Definition.\\
\texttt{\{}\\
\qquad \texttt{int sum = 0;}\\
\qquad \texttt{for (int i = 0; i < size; i++)}\\
\qquad \qquad \texttt{sum += (*foo)(arr[i]);}\\
\texttt{\}}

Here \texttt{foo} is a pointer to a function and \texttt{(*foo)} is the function.\\

\texttt{sum\_f(size, arr, square)} // Usage.Sum of squares.\\
\texttt{sum\_f(size, arr, cube)} // Usage.Sum of cubes.\\

NOTE: Function names act as pointers to the function. \\


\vfill \null
\columnbreak


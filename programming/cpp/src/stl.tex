\section{Data structures}

\subsection{Pair}
Eg: \texttt{pair<int, int> x = {1,2};}\\
\begin{tabularx}{\linewidth}{lX}
\texttt{x.first} & Returns 1. \\
\texttt{x.second} & Returns 2.
\end{tabularx}

\subsection{Tuple}
Fixed length collection of heterogenous values.\\
Eg: \texttt{tuple<int, string, double> t;}\\

\textbf{Operations related to tuple}\\
\begin{tabularx}{\linewidth}{lX}
\texttt{get<i>} & Return $i^th$ value of t. Eg: \texttt{get<0>(t);}.\\
\texttt{make\_tuple()} & Returns tuple out of individual values.\\
& Eg: \texttt{make\_tuple(5, "Paul", 23.4)}.\\
\texttt{tie()} & Unpacking tuple.\\
& Eg: \\
& \texttt{tie(n, s, x) = t;}\\
& \texttt{tie(n, std::ignore, x) = t;}\\
\end{tabularx}

\subsection{Vectors}

\textbf{Definiton:}\\
\texttt{vector<T> v;} 

\textbf{Element access}

\begin{tabularx}{\linewidth}{lX}
\texttt{operator[]}	& \\
\texttt{v.back()} & Last element \\
\texttt{v.front()} & First element \\
\end{tabularx}


\textbf{Modifiers}

\begin{tabularx}{\linewidth}{lX}
\texttt{v.push\_back(5)} & Add \texttt{5} to the vector. Simillar to push in stack data structure.\\
\texttt{v.insert()} & Inserts elements / range in \texttt{v}.\\
& Eg: Append \texttt{u} to \texttt{v}.\\
& \texttt{v.insert(v.end(), u.begin(), u.end());}\\
\texttt{v.emplace} & \texttt{v.emplace\{iterator pos, Args\}}. New element is constructed in place using \texttt{Args} and inserted at pos.\\
\texttt{v.emplace\_back} & \texttt{v.emplace\_back\{Args\}}. New element is constructed in place using \texttt{Args} and inserted at the end.\\
& \textcolor{red}{IMPORTANT: Note the differences between pushback, insert, emplace and emplaceback}.\\
\texttt{v.pop\_back()} & Remove element on top of the stack.\\
\texttt{v.erase()} & Remove element / range from \texttt{v}.\\
\texttt{v.clear()} & Clears all the elements in \texttt{v}.\\
\end{tabularx}

\textbf{Capacity}

\begin{tabularx}{\linewidth}{lX}
\texttt{v.size()} & Returns the size of the vector.\\
\texttt{v.capacity()} & Current memory allocated to \texttt{v} in terms of number of elements.\\
\texttt{v.max\_size()} & Maximum memory available for \texttt{v}.\\
\texttt{v.reserve()} & Reserve a minimum size for v. Does not affect v.size().\\
\texttt{v.shrink\_to\_fit()} & Shrink the capacity to fit the size.\\
\texttt{v.empty()} & Test if \texttt{v} is empty.\\
\end{tabularx}

\vspace*{8pt}
\texttt{tie} Can be used to unpack a vector.\\
\textit{Example:} \texttt{tie(a, b, c) = v;}\\


\vfill \null
\columnbreak

\subsection{String}

A \texttt{string} is like a vector of characters.

\textbf{Definition:}\\
\texttt{string s = "Hello";}

\textbf{String operations:}\\
\begin{tabularx}{\linewidth}{lX}
\texttt{a + b} & Concatenate.\\
\texttt{s.append(t)} & append \texttt{t} to \texttt{s}.\\
\texttt{s.compare(t)} & returns \texttt{0} if \texttt{s == t}, returns $< 0$ if $s \prec t$, else returns $> 0$.\\
\texttt{s.substr(start, len)} & Select a substring. \\
\texttt{s.find(str2)} & find first occurence of \texttt{str2}. \\
\texttt{s.find(str2, pos)} & Find first occurence of \texttt{str2} by searching from the positon \texttt{pos}.
\end{tabularx}

\subsection{Set}
\textbf{Definition}\\
\texttt{set<type> s;}\\
\texttt{multiset<type> s;}\\
\texttt{unordered\_set<type> s;}\\
\texttt{unordered\_multiset<type, hasher> s;}\\
\texttt{hasher}: Refer Custom Hash Function.\\


\textbf{Set operations}\\
\begin{tabularx}{\linewidth}{lX}
\texttt{s.insert(item);} & \\
\texttt{s.erase(item);} & \\
\texttt{s.count(item);} & \\
\texttt{s.find(item);} & Returns the iterator that points to \texttt{item}.\\
\texttt{s.insert()} & Inserts elements in s.\\
& Eg: Insert elements from t to s.\\
& \texttt{v.insert(v.end(), u.begin(), u.end());}\\
\texttt{s.size()} & \\
& Not availabe for \texttt{unordered\_set}  and \texttt{unordered\_multiset}.\\
\end{tabularx}

\begin{tabularx}{\linewidth}{lX}
\texttt{s.lower\_bound(x)} & Returns iterator to the smallest element that is larger than or equal to x.\\
\texttt{s.upper\_bound(x)} & Returns iterator to the smallest element that is larger than x.\\
\end{tabularx}

\textcolor{red}{NOTE: erase method remove all instances of the item in the multiset.}

\subsection{Bit set}

\textbf{Definiton}\\
\texttt{bitset<size> s;}\\
Examples:\\
\texttt{bitset<5> s;}
\texttt{bitset<5> s(string("1100110"));}\\
\textbf{Operations on Bit set:}\\
\begin{tabularx}{\linewidth}{lX}
\texttt{a \& b} & Bitwise AND.\\
\texttt{a | b} & Bitwise OR.\\
\texttt{a \^ b} & Bitwise EXOR.\\
\texttt{~a} & Bitwise negation.\\
\end{tabularx}

\subsection{Map}

\texttt{map} : Balanced binary tree.\\
\texttt{unordered\_map}: Hash list.\\

\textbf{Definition:}\\
\texttt{map<type1, type2> m;}\\
Eg: \texttt{map<string, int> m;}\\
\texttt{unordered\_map<type1, type2, hasher> m;}\\
\texttt{hasher}: Refer Custom Hash Function.\\

\textbf{Operations on map}\\
\begin{tabularx}{\linewidth}{lX}
\texttt{m["banana"]} & Retrive value associated with \texttt{banana}.\\
\texttt{m.count("banana")} & Return 1 if the key is present else 0.\\
\texttt{m.size();} & \\
\end{tabularx}

\subsection{Stack}
Eg: \texttt{stack<int> s;}\\

\begin{tabularx}{\linewidth}{lX}
s.push(5) & \\
s.top() & \\
s.pop() & \\
\end{tabularx}

\subsection{Queue}
Eg: \texttt{queue<int> q;}\\

\begin{tabularx}{\linewidth}{lX}
\texttt{q.push(5)} & Adds element to the end.\\
\texttt{q.pop()} & Removes first element.\\
\texttt{q.front()} & Returns first element.\\
\texttt{q.empty()} & True if q is empty.\\
\end{tabularx}

\subsection{Dequeue}
Eg: \texttt{dequeue<int> d;}\\

\begin{tabularx}{\linewidth}{lX}
\texttt{d.push\_back(5)} & Add to the top.\\
\texttt{d.push\_front(2)} & Add to the bottom.\\
\texttt{d.pop\_back()} & Remove from the top.\\
\texttt{d.pop\_front()} & Remove from the bottom.\\
\end{tabularx}


\subsection{Heap}
\textbf{Format:}\\
\textbf{Constructing a heap:}\\

\vspace{-6pt}

\begin{verbatim}
make_heap(iterator f, iterator s);
make_heap(iterator f, iterator s, comp);
\end{verbatim}

\vspace{-12pt}

\begin{verbatim}
pop_heap(iterator f, iterator s);
pop_heap(iterator f, iterator s, comp);
\end{verbatim}

\vspace{-12pt}

\begin{verbatim}
push_heap(iterator f, iterator s);
push_heap(iterator f, iterator s, comp);
\end{verbatim}

NOTE:\\
\texttt{f} points to the first element and \texttt{s} points next to the second element.\\
\texttt{f} and \texttt{s} can be pointers when the heap is constructed over a array.

\textbf{Compare function:}\\

\vspace{-6pt}

\begin{verbatim}
bool compare(x, y){
	return(x < y); // For max heap
	return(x > y); // For min heap
}
\end{verbatim}

\vfill\null
\columnbreak

\begin{minipage}{\linewidth}

\subsection{Priority queue}
\textbf{Format:}\\
\texttt{priority\_queue(type, vector<type>, compare)}\\
\textbf{Max priority queue [Default]}\\
Eg: \texttt{priority\_queue<int> q;}\\
\textbf{Min priority queue}\\
\texttt{priority\_queue<int, vector<int>, greater<int>> q;}

\begin{tabularx}{\linewidth}{lX}
\texttt{q.push(3)} & \\
\texttt{q.top()} & Returns the largest element.\\
\texttt{q.pop()} & Removes the largest element.
\end{tabularx}

\vspace{6pt}

\textbf{Compare function for priority queues}


\begin{verbatim}
class compare{
public:
    bool operator()(const T& x, const T& y) const {
        return(x < y); // Max priority queue.
        // return(x > y); // Min priority queue.
    }
}
\end{verbatim}

\subsection{Iterators}
Iterators are like pointers but more specific to a collection.\\
Not applicable for \texttt{unordered\_set}, \texttt{unordered\_map} and \texttt{priority\_queue}.\\

Example definition: \\
\texttt{vector<int>::iterator it;}\\
\texttt{set<string>::iterator it;}\\

\begin{tabularx}{\linewidth}{lX}
\texttt{s.begin()} & Returns iterator pointing to the first element in the collection.\\
\texttt{s.end()} & Returns iterator pointing next to the last element in the collection.
\end{tabularx}

\textbf{Iterating over a map:}\\
Here \texttt{*it} is a \texttt{pair} of the key and value.
Eg: \texttt{pair<string, int> p = *it;}\\
Eg:
\begin{verbatim}
    map<string, int>::iterator it = m.begin();
    // Accessing key and value
    string key = it->first;
    int value = it->second;
\end{verbatim}

\textbf{Reverse Iterators}

\subsection{Representing graphs}

\subsection{Adjacency list}
Eg:\\
\texttt{vector<int> adj[N]}// Unweighted graph.\\
\texttt{vector<pair<int,int>> adj[N]}// Weighted graph.\\

\subsection{Adjacency matrix}

Eg: \texttt{int adj[N][N];}

\subsection{Edge list}

Eg:\\
\texttt{vector<pair<int,int>> edges}Unweighted graph.\\
\texttt{vector<tuple<int,int,int>> edges} // Weighted graph.\\

\end{minipage}

\section{Previous Mistakes:}

\subsection{Using proper variable types}
Eg: using \texttt{int} instead of \texttt{long long int} would cause erroneous output.\\

\subsection{Be careful of unsigned int}
Eg
\vspace{-6pt}
\begin{verbatim}
unsigned x; cin >> x;
while(x >= 0){
    // Do something
}
\end{verbatim}
The above will never terminate, since \texttt{x} never becomes negative.

\subsection{Upper bound and lower bound}
Check for edge cases. While searching for \texttt{x}, if \texttt{x} is not found, then \texttt{lower\_bound} points to an element larger than \texttt{x}. The following might be helpful.


\begin{verbatim}
if(lower_bound > begin && *lower_bound > x) lower_bound--;
if(lower_bound == end) lower_bound--;
if(upper_bound == end) upper_bound--; // Might be 
necessary sometimes.
\end{verbatim}


\subsection{Loops}
\begin{itemize}
\item  Using variable in boundary condition. Eg:\\
\begin{verbatim}
// value of x changes within the loop. 
for(int i = 0; i < (d[x] - c); i++) // Dangerous !!

k = d[x] - c;
for(int i = 0; i < k; i++) // 
\end{verbatim}

\item When the variable in boundary condition is updated by mulitiplication.\\
In the following loop when n $>$ 1 and a = 1, this loop won't terminate.
\begin{verbatim}
    int x = 1;
    while(x < n){
        x = a * x; 
    }
\end{verbatim}

\end{itemize}

\subsection{Bugs in compiler directives}
In Bioinformatics contest 2021, a bug was because I use \texttt{bitset<M>}, where \texttt{M} was specified as compiler directive. While debugging I did not pay attention to \texttt{\#define} statements.\\ 
\textit{Lesson:} While debugging pay attention to the entire code, particularly compiler directives. They are particulary important because they are hidden and not directly noticeable in the code.

\subsection{Bugs due to augmented data}
\textit{Ref:} ABC\_207 problem C.
There were two vectors, \texttt{v} and \texttt{t} associated by index. The vector \texttt{v[i]} corresponds to \texttt{t[i]}. In this problem, I sorted v independently from t. Thus this broke the association between \texttt{v} and \texttt{t} resulting in erroneous results.\\

\subsection{min\_element and max\_element}
Do not use these two functions to find minimum and maximum elements in an unsorted array. Read documentation to know more about them. 


\subsection{map vs multiset, NOT exactly a mistake}
In one problem I used multiset to keep count of some values. It was very slow compared to using map to keep count of those values.

\subsection{Math errors}
Division by zero and logarithm of zero.

\subsection{Not accessing out of bound elements}
\vspace{6pt}
\begin{verbatim}
if(a[i] == 0) {do something;} // Wrong.
\end{verbatim}

\vspace{-14pt}

\begin{verbatim}
if(i < n && a[i] == 0){do something;}// Correct.
\end{verbatim}

\subsection{Order of precedence and brackets}
Example:(CRF-732: 1546C)
\begin{verbatim}
if(l_pos[k] - l_pos[k-1] % 2 == 1){do something}// Wrong;
if((l_pos[k] - l_pos[k-1]) % 2 == 1){do something}// Correct;
\end{verbatim}

\subsection{Using braces appropriately in conditionals and loops}
Example:(CRF-732: 1546C)
\begin{verbatim}
// Wrong if do_2 depends on x :
if(x) do_1;
    do_2;
\end{verbatim}


\begin{verbatim}
// Correct if do_2 depends on x :
if(x){ do_1;
    do_2;
}
\end{verbatim}


\subsection{Errors due to 0-based and 1-based indexing}
The data in most programming contests use 1-based indexing. Error could arise while reading or using such data.\\
Eg:
\begin{verbatim}
	for(int i = 0; i < n; i++){
		int temp; cin >> temp;
		a[temp]++;
	}
\end{verbatim}

Here if a is 0-based and input temp is 1-based, then error will arise. The following is correct.

\begin{verbatim}
	for(int i = 0; i < n; i++){
		int temp; cin >> temp;
		a[temp-1]++;
	}
\end{verbatim}

\subsection{Error while using obj.size() as limit in a loop}
Obj.size() return a value of type \texttt{size\_t} which (probably) unsigned long long int. Therefore it inherits the same dangers associated with using \texttt{unsigned int}(13.2).
Eg: Here if v.size() is zero, v.size()-1 will not return -1, rather it would return the largest \texttt{long long}.
\begin{verbatim}
    for(int i = 0; i <= v.size()-1; i++){}
\end{verbatim}
\vspace{-6pt}
Probable a correct way to write the above code is as follows:
\vspace{-6pt}
\begin{verbatim}
    for(int i = 0; i <= (int)v.size()-1; i++){}
\end{verbatim}
\vspace{-6pt}
\textbf{Another example:}
\vspace{-6pt}
\begin{verbatim}
    for(int i = r.length()-2; i >=0; i++){}	
\end{verbatim}
\vspace{-6pt}
Correct version.
\vspace{-6pt}
\begin{verbatim}
    int rl = r.length();
    for(int i = rl-2; i >=0; i++){}	
\end{verbatim}

\subsection{Loop condition dependent on string::find}
Bugs could arise because \texttt{string::find} return -1 when the query is not found.\\
Eg: 
\vspace{-6pt}
\begin{verbatim}
    for(int i = 0; i < size; i++){
        i = s.find(c, i);
        //More code...
    }
\end{verbatim}


\vspace{-6pt}
Correct version:\\

\vspace{-6pt}
\begin{verbatim}
    for(int i = 0; i < size; i++){
        i = s.find(c, i);
        if(i == -1) break;
        //More code...
    }
\end{verbatim}

\subsection{Semicolon typo}
Eg: 
\begin{verbatim}
for(int i = 0; i < n; i++);{
    cout << i << '\n';
}
\end{verbatim}
In above code the for loop is prematurely terminated. This would result in runtime error.

\subsection{Bugs due to using infinity}{
Eg:\\
\begin{verbatim}
const int inf = numerics::limits<int>max();
int x = inf;
cout << x + inf << '\n';
\end{verbatim}
}
The output here would not be inf. I made a simillar mistake while using infinity in Floyd-Warshall's algorithm.

\subsection{No code after return}

Once I made the following mistake:

\begin{verbatim}
	return r.top(); r.pop();	
\end{verbatim}
Obviously, that \verb!r.pop()! did not work.

\subsection{Beware when using logical operators with functions}

The following is dangerous:

\begin{verbatim}
	ans = ans && f(x);
\end{verbatim}

If \texttt{ans} is initally false, f won't be executed.	

Better one would be (I am not sure if this always works with all programming languages):

\begin{verbatim}
	ans = f(x) && ans; 
\end{verbatim}

I made this mistake in \textcolor{red}{LeetCode 130. Surrounded Regions}.

\subsection{Be careful with char vs int}
I made the following mistake:\\
\begin{verbatim}
	if(mat[i][j] == 1){
		// do something.
	}
\end{verbatim}
Here \verb!mat! was \verb!vector<vector<char>>! I should have tested for \verb!mat[i][j] == '1'!.

\subsection{Be careful about pass by reference.}
Ref: A student's code. \\
Not using pass by reference  resulted in failure in the update of value of the target.
This caused issue in an implementation of linked-list.

\subsection{a = a + 1 vs a += 1}
Ref: Discussion with a labmate.
This is in the context of python programming.
a = a + 1 creates a new object and assigns it to a.
a += 1 does not create a new object. It modifies the existing object.

\subsection{Failure to initialize all the variables}
Ref: A student's code.\\
The student did not initialize all the variables. This resulted in a bug.
This was particularly critical because one  of the class variable was a pointer.
This resulted in segmentation fault. It also cause TLE, but I am not sure about the exact mechanism.

\vfill \null
\columnbreak


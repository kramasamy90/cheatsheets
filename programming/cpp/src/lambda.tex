\section{Lambda Functions}

\textit{Format:}\\
\texttt{[capture](parameters) -> return\_type \{body\}}\\

\begin{itemize}
    \item \textbf{Capture:} Captures variable from the surrounding scope by value or reference.\\
        Eg: \texttt{[a,\&b]}\\
    \item \textbf{Parameters:} Parameters of the function.\\
        Eg: \texttt{(int a, int b)}\\
\textit{Return type:} Return type of the function.\\
\end{itemize}

\subsubsection{Examples}
\textbf{Sorting in reverse order:}
\begin{verbatim}
    sort(v.begin(), v.end(), [](int a, int b) -> bool {
        return a > b;
    });
\end{verbatim} 

\textbf{Counting number of elements in a vector:}
\begin{verbatim}
    int count = count_if(v.begin(), v.end(), [](int a) -> bool {
        return a > 5;
    });
\end{verbatim}

\vfill\null
\columnbreak

\textbf{Using capture:}\\
Lambda functions can also be stored in variables like the add in the following example.\\
\begin{verbatim}
   #include <iostream>

    int main() {
        int x = 10;
        int y = 20;
        auto add = [x, y]() -> int {
            return x + y;
        int result = add();
    };
}
 
\end{verbatim}
\section{Input output}

\subsection{File Access}

Format:\\

\texttt{FILE *fp;}\\
\texttt{FILE *fopen(char *name, char *mode);}\\
\texttt{int fclose(FILE *fp);}\\
\texttt{fp = fopen(name, mode);}\\

Allowable modes include:\\
\begin{tabularx}{\linewidth}{lX}
\texttt{"r"} & read.\\
\texttt{"w"} & write.\\
\texttt{"a"} & append.\\
\texttt{"b"} & binary files: \texttt{"rb"}, \texttt{"wb"}, \texttt{"ab"}.\\
\end{tabularx}
NOTE: \textcolor{red}{Difference between \texttt{rb} and \texttt{r} etc:}\\
While using just \texttt{"r"} might work in writing binary files, it might cause trouble due misinterpretation of special characters like LF and CR. See: \url{https://stackoverflow.com/q/2174889/5607735} 

\subsubsection{Standard I/O}
Can be treated in the same way as file pointer.\\
\texttt{stdin}, \texttt{stdout}, \texttt{stderr}\\

\subsection{Reading and writing a file}
\begin{verbatim}
size_t fread{const void* ptr, size_t size,/
size_t count, FILE* stream};

size_ t fwrite{const void* ptr, size_t size,/
size_t count, FILE* stream};
\end{verbatim}

\subsection{Character I/O}
\begin{tabularx}{\linewidth}{lX}
\texttt{getchar} & Takes input from standard input.\\
& \texttt{int getchar()}\\
& Equivalent to \texttt{getc(stdin)}\\
\texttt{putchar} & Output to standard output.\\
& \texttt{int putchar(int)}\\
& Equivalent to \texttt{putc(c), stdout}\\
\texttt{getc} & \texttt{int getc(FILE *fp)}, Eg: \texttt{getc(stdin)}\\
\texttt{putc} & \texttt{int putc(int c, FILE *fp)}\\
\texttt{ungetc} & \\
\end{tabularx}

\subsection{Line I/O}
\begin{tabularx}{\linewidth}{lX}
\texttt{gets} & Reads until \texttt{EOF}.\\
& \texttt{gets} deletes terminal \textbackslash \texttt{n}\\
\texttt{puts} & Writes line to stdout.\\
& \texttt{puts} adds terminal \textbackslash \texttt{n}\\
\texttt{fgets} & \texttt{char *fgets(char *line, int maxline, FILE *fp)}\\
\texttt{fputs} & \texttt{int *fputs(char *line, FILE *fp)}\\
\texttt{getline} & Eg: \texttt{getline(cin, s)}. Where \texttt{s} is a string.
\end{tabularx}

\begin{mdframed}
\textbf{\textcolor{red}{NOTE: Never use \texttt{gets}}}\\
\texttt{gets} does not check for buffer overrun, and keeps reading until it encouters new line or \texttt{EOF}.
\textcolor{red}{It has been used to break computer security.}
\end{mdframed}

\vfill \null

\subsection{\texttt{printf} and \texttt{scanf}}
\subsubsection{Conversion formats for \texttt{printf} and \texttt{scanf}}
\begin{tabularx}{\linewidth}{c|X}
\hline 
\textbf{Character} & \textbf{Argument type; Printed as}\\
\hline
\texttt{d,i} & int; decimal number\\
\texttt{o} & int; unsigned octal number (without leading zero)\\
\texttt{x,X} & int: unsigned hexadecimal number (without a leading 0x or 0X), using \texttt{abcdef} or \texttt{ABCDEF}.\\
\texttt{u} & int; unsigned decimal number.\\
\texttt{zu} & \texttt{size\_t}.\\
\texttt{c} & int; single character.\\
\texttt{s} & char *; Print string, scan word.\\
& \textcolor{red}{NOTE: \texttt{scanf} reads only a word and not the entire string.}\\
\texttt{f} & double; [-]\textit{m.dddddd}.\\
\texttt{e,E} & double; [-]\textit{m.dddddd}$\pm$\textit{xx}, or [-]\textit{m.dddddd}$\pm$\textit{xx}.\\
\texttt{g,G} & double; \%e or \%E if exponent is $<$ -4 or $>=$ precision, else use \%f.\\
\texttt{p} & void *; pointer.\\
\texttt{\%} & no argument; Print \%.\\
\hline
\end{tabularx}

Between \% and conversion character there may be in order:\\
\begin{itemize}
\item Minus sign: Left justification.
\item Number: Minimum field width.
\item Period: Separates field width from precision.
\item Number: Precision. \\
	For float: number of digits after decimal.\\
	For int: minimum number of digits.\\
	For string: maximum number of characters to be printed.\\ 
\item h: for short integer.
	l: for long integer.\\
\end{itemize}


\subsubsection{Formatting rules}
Examples
\begin{tabularx}{\linewidth}{lX}
& \textbf{Wildcard}\\
\textcolor{red}{\%*d} & Here precision can be specified dynamically. Eg: \\
& \texttt{printf("\%*d", 6, foo);}, this is equivalent to \texttt{printf("\%6d",foo);}\\
& \textbf{Integers}\\
\%d & print as decimal integer.\\
\%6d & print as decimal integer, at least 6 characters wide.\\
& \textbf{Floating point numbers}\\
\%6f & print as floating point.\\
\%.2f & print as floating point, 2 characters after decimal point.\\
\%6.2f & print as floating point, at least 6 characters wide and 2 characters after decimal point.\\
& \textbf{String:} example \texttt{hello, world} (12 chars).\\
\texttt{:\%s:} & \texttt{:hello,\textvisiblespace world:}\\
\texttt{:\%10s:} & \texttt{:hello,\textvisiblespace world:}\\
\texttt{:\%.10s:} & \texttt{:hello,\textvisiblespace wor:}\\
\texttt{:\%-10s:} & \texttt{:hello,\textvisiblespace world:}\\
\texttt{:\%.15s:} & \texttt{:hello,\textvisiblespace world:}\\
\texttt{:\%-15s:} & \texttt{:hello, world\textvisiblespace\textvisiblespace\textvisiblespace:} \\
\texttt{:\%15 .10s:} & \texttt{:\textvisiblespace\textvisiblespace\textvisiblespace\textvisiblespace\textvisiblespace hello, wor:}\\
\texttt{:\%-15 .10s:} & \texttt{:hello,\textvisiblespace wor\textvisiblespace\textvisiblespace\textvisiblespace\textvisiblespace\textvisiblespace:}\\
\end{tabularx}

\subsubsection{Printf and variants}

\begin{itemize}
\item \texttt{int printf(char *format, arg1, arg2, $\cdots$);}\\
Print output to stdout.
\item \texttt{int sprintf(char *string, char *format arg1, arg2, $\cdots$);}\\
Print output to \texttt{string}.
\item \texttt{int fprintf(FILE *fp, char *format arg1, arg2, $\cdots$);}\\
Print output to file pointed by the file pointer \texttt{fp}.
\end{itemize}

\textbf{Example}:\\
\texttt{printf("\%s\textbackslash n", string\_var);}\\
\texttt{printf("\%s", "hello, world");}\\
\texttt{printf("square of n is \%d\textbackslash n", n\_square);}\\

\subsubsection{Scanf and variants}

\begin{itemize}
\item \texttt{int scanf(char *format, arg1, arg2, $\cdots$);}\\
Scan input from \texttt{stdin}.
\item \texttt{int sscanf(char *string, char *format arg1, arg2, $\cdots$);}\\
Scan input from string.
\item \texttt{int fscanf(FILE *fp, char *format arg1, arg2, $\cdots$);}\\
Scan input from the file pointed by the file pointer \texttt{fp}.
\end{itemize}

\textcolor{red}{NOTE1: unlike printf, in scanf the variable are pointers.}\\
\textcolor{red}{NOTE2: In \texttt{scanf} \texttt{\%s} reads a word and not the entire string.}\\
\textcolor{red}{Also see:} \textcolor{blue}{\url{https://stackoverflow.com/a/1248017/5607735}}\\

\textbf{Examples:}
\vspace{-6pt}
\begin{verbatim}
scanf("%d", &n);	
sscanf(string, "%d", &n);	
fscanf(fp, "%d", &n);	
\end{verbatim}

\subsection{\texttt{cin} and \texttt{cout}}

	\begin{tabularx}{\linewidth}{lX}
		\texttt{cin} & Read input until space, or tab or LF.\\
		\texttt{cout} & Output.\\
	\end{tabularx}
\vspace{-8pt}
	\begin{mdframed}[backgroundcolor=blue!20]
		\textbf{NOTE:}\\
			\textbullet\ Use the following to increase speed ot I/O.
\vspace{-8pt}
			\begin{verbatim}
				ios_base::sync_with_stdio(false);
			cin.tie(nullptr); cout.tie(nullptr);	
		\end{verbatim}
\vspace{-8pt}
		\textbullet\ Use \texttt{cin.ignore()} after \texttt{cin} and before using \texttt{getline}.
	\end{mdframed}

\subsection{Miscellaneous}

\texttt{int fflush(FILE *stream)}\\
Causes any buffered but unwritten data to be written. The effect is undefined on an input stream.\\
\texttt{int fclose(FILE *stream)}\\
Flushes any unwritten data for \texttt{stream}, discardes any unread buffered input.

\vfill\null
\columnbreak



\section{Arithemetic and logical operation}

    \begin{tabularx}{\linewidth}{l|lX}
        \hline\\
        & Operators & Associativity \\
        \hline \\
        1 & \texttt{() [] -$>$} \textbf{.} & left to right\\
        2 & \texttt{! $\tilde{}$ ++ -- + - * \& (type) sizeof} & right to left\\
        3 & \texttt{* / \%}(reminder) & left to right\\
        4 & \texttt{+ -} & left to right \\
        5 & \texttt{$<<$ $>>$} & left to right \\
        6 & \texttt{$<$ $<=$ $>$ $>=$}  & left to right\\
        7 & \texttt{== !=} & left to right\\
        8 & \texttt{\&} & left to right \\
        9 & \texttt{\^} & left to right \\
        10 & \texttt{|} & left to right\\
        11 & \texttt{\&\&} & left to right\\
        12 & \texttt{||} & left to right \\
        13 & \texttt{?:} & left to right \\
        14 & \texttt{= += -= *= /= \%= \&= \^{}= |= $<<$= $>>$=} & right to left\\
        15 & \texttt{,} & left to right\\
        \hline
    \end{tabularx}

    \subsection{Explanation for selected operators}

        \begin{itemize}
            \item \texttt{->} and \textbf{.}\\
            See \textbf{Structures}.
            \item \textbf{Casting}\\
            Changes the type of a variable.\\
            Eg:  \texttt{float a = (int) 3/2;}// \texttt{a} is 1.0.\\
            \item \textbf{sizeof}\\
            \texttt{sizeof num;} // Give size of the variable num.\\
            \texttt{sizeof (int)} // Gives size of int, i.e. 4.\\
            \item \textbf{Bitwise operators}
            \begin{itemize}
            \item \texttt{>>}\\
            Eg: \texttt{a = b >> 1} //Shift \texttt{b} bitwise to the right by 1 bit.\\
            Fill left most bits with the original left most bit.
            \item \texttt{<<}\\
            Eg: \texttt{a = b << n} //Shift \texttt{b} bitwise to the left by n bits.\\
            Fill right most bits with zero.
        \end{itemize}

        Others:\\
        \begin{tabularx}{\linewidth}{l|X}
            \hline
            \texttt{\&} & Bitwise AND\\
            \texttt{$\mid$} & Bitwise inclusive OR\\
            \texttt{\^} & Bitwise EXOR\\
            \texttt{\~} & Bitwise NOT\\
            \hline
        \end{tabularx}


        \item \textbf{Comma operator}
        \begin{itemize}
            \item \texttt{expr1,expr2}\\
            Use sparingly, Eg: \texttt{i++,j--}\\
            \item \texttt{x = (Conditional expresion)? \textit{expr1}: \textit{expr2}}\\
            \texttt{x =expr1} if \texttt{Conditional expression} is true, else \texttt{x = expr2}.
        \end{itemize}

        \end{itemize}


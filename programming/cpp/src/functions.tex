\section{Functions}

\subsection{\texttt{int main} function: command line arguments}

\texttt{main(int argc, char *argv[])}\\
\texttt{\{}\\
\qquad \texttt{$\cdots$}\\
\texttt{\}}

\begin{tabularx}{\linewidth}{l|X}
\texttt{argc} & Number of arguments including the program name. \\
\texttt{argv[0]} & Pointer to program name.\\
& \textcolor{red}{NOTE: \texttt{argv[0] represents name of the compiled program and not the source file and how it is called eg: \texttt{a.out} vs \texttt{./a.out}}}\\
\texttt{argv[i]} & Pointer to i$^{th}$ argument.\\
\texttt{argv[argc - 1]} & Pointer to last argument. \\
\end{tabularx}




\subsection{Function definition}
\texttt{\textit{type function\_name}(\textit{argument list})}\\
\texttt{\{}\\
\qquad \texttt{\textit{statements}}\\
\qquad \texttt{$ \cdots $}\\
\qquad \texttt{return \textit{expression}}\\
\texttt{\}}

eg:\\

\texttt{int foo(int num, int cars[],char *pn)}\\
\texttt{\{}\\
\qquad \texttt{\textit{statements}}\\
\qquad \texttt{$ \cdots $}\\
\qquad \texttt{return \textit{expression}}\\
\texttt{\}}

\subsection{Function declaration}

\textbf{Implicit declaration}:\\
The function is assumed to return \texttt{int}. Nothing is assumed about the arguments.\\
\textbf{Explicit declaration}:\\
Examples:\\
\texttt{int foo(int, int [], int *);}\\
\texttt{int foo(int x, int y[], int *p);}\\
Explicit declaration is not necessary if the function is defined before \texttt{main()}.\\

\subsection{Passing functions as arguments}
Example: passing \texttt{foo2} to \texttt{foo}.\\
\textbf{Declaration:}\\
\texttt{int foo(int x, int (*foo2)(int y));}\\
\textbf{Usage:}\\
\texttt{foo(x, foo2);}\\
\textcolor{red}{Also see:}\textit{7.7 Pointers to functions} 
\vfill \null
\columnbreak


\section{Networking}

\subsection{Basic Tools}

\begin{itemize}
\item \textbf{\texttt{ifconfig}}. Lists network properties like IP addresses.\\
        Lists info regarding loop-back, ethernet and WiFi (etc.)\\
        \begin{itemize}
        \item \texttt{lo} is loop back.\\
        \item \texttt{inet}: IPv4.\\
        \item \texttt{inet6}: IPv6.\\
        \item \texttt{ether}: MAC address.\\
        \item \texttt{RX}: Stats on received data.\\
        \item \texttt{TX}: Stats on transmitted (sent) data.\\
        \end{itemize}
\item \texttt{ip addr show}. Simillar to ifconfig. Show only the IP addresses and not the stats.
\item \texttt{ping}. Eg: \texttt{ping google.com}\\
        Give TTL, RTT and packet loss when connecting to the host.\\
\item \texttt{hostname}
\end{itemize}


\subsection{Cryptography}

\subsubsection{SSH}

\begin{itemize}
    \item \textbf{Generating the keys.} Examples:\\
            \texttt{ssh-keygen -t ed25519 -C "your\_email@example.com"}\\
            \texttt{ssh-keygen -t rsa -b 4096 -C "your\_email@example.com"}\\
            \texttt{-t}: Algorithm to use.\\
            \texttt{-b}: Key length.\\
            \texttt{-C}: Label for the key, generally email ID.\\
    \item   Public keys: \texttt{$\sim$/.ssh/id\_rsa}\\
    \item   Private keys: \texttt{$\sim$/.ssh/id\_rsa.pub}\\
\end{itemize}

\subsubsection{GPG Keys}

\begin{itemize}
\item \textbf{Genertate key:} \texttt{gpg --full-generate-key}\\
        Follow the instructions to generate the key.\\
\item \texttt{gpg --list-keys}.
\item  \textbf{Export public key:}\\
        \texttt{gpg --armor --export your\_email\@example.com > public\_key.asc}\\
            \texttt{--armor} Output key in ASCII format.\\
            \texttt{--export} Specify the key.\\
\item \textbf{Import key:}\\
        \texttt{gpg --import public\_key.asc}\\
\item \textbf{Sign a file:}\\
        gpg --sign file.txt\\
\item  \textbf{Encryption / Decryption:}\\
        \textit{Encrypt:}\\
        \texttt{gpg --encrypt --recipient recipient\_email\@example.com file.txt}\\
        \textit{Decrypt:}\\
        \texttt{gpg --decrypt file.txt}\\
\item Keys are located in \texttt{$\sim$/.gnupg} directory.
\end{itemize}


\vfill\null
\columnbreak


\subsection{File Transfer}

\begin{tabularx}{\linewidth}{lX}
\hline
\texttt{wget \textit{url}} & Download file from http or ftp.\\
\texttt{-- accept, -A "..."} & Only download files matching this criteria. Eg "*.fastq"\\
\texttt{-- reject, -R} & Similar to above\\
\texttt{--no-directory, -nd} & Don't download directory structure.Only files.\\
\texttt{--recursive, -r} & \\
\texttt{--no-parent, -np} & Don't move above parent directory. This is important to avoid downloading unnecessary data.\\
\texttt{-O} & Output filename.\\
\texttt{-e robots=off} & To not want wget to follow '\texttt{robot.txt}'.\\
& See: \href{https://stackoverflow.com/a/11124664}{This answer} \\
\hline
\end{tabularx}

\textcolor{red}{Other options: --limit-rate, --user=user, --ask-password}\\


\begin{tabularx}{\linewidth}{lX}
\hline
\texttt{curl \textit{url} > file} & Redirect output to file.\\
\texttt{curl -O <file>} & download to file.\\
\texttt{-L,--location} & Download ultimate page and not the redirect page.\\
\hline
\end{tabularx}
Curl can also download form SFTP and SCP. Also checkout RCurl and pycurl.


\vfill \null
\columnbreak

\subsection{SSH}
\begin{itemize}
\item \textbf{Usage}\\
\texttt{\$ ssh host}\\
\texttt{\$ password:}\\

\item \textbf{Examples of host}\\
\texttt{192.162.82.120}\\
\texttt{bioclust.myuniversity.edu}\\
\texttt{darwin@192.162.82.120}\\
\texttt{darwin@bio.univ.edu}\\

\item \textbf{Options} \\

\texttt{-v} verbose. Verbosity can be increased by: \texttt{-vv} or \texttt{-vvv}. \\
\texttt{-p} port. Eg: \texttt{ssh -p 5043 cdarwin@bio.univ.edu}\\
\qquad Default port is \texttt{22}

 
\item \textbf{Using alias:} To use alias create the file \texttt{\~{}/.ssh/config} and store server as info as below. 
\texttt{Host bio\_serv}\\
\qquad \texttt{HostName 190.512.171.29}\\
\qquad \texttt{User cdarwin}\\
\qquad \texttt{Port 50434}\\
Also applies for \texttt{Rsync and scp}\\

\item \textbf{SSH keys}: SSH key to connect without password. Eg:\\
\texttt{\$ ssh-keygen -b 2048}\\
This command request the following:\\
\begin{itemize}
\item File to save the key. By default this is:\\
\texttt{/Users/username/.ssh/id\_rsa}
NOTE: This file is the private key.\\
\item Passphrase:\\
Not necessary but good to use.\\
\end{itemize}
Private key: \texttt{~/.ssh/id\_rsa}\\
Public key: \texttt{~/.ssh/id\_rsa.pub}\\

\texttt{\$ chmod 400 id\_rsa \# restrict access to private key}\\
\texttt{\$ ssh-add} \\

\end{itemize}


\subsection{Setting up a server}
Use "Open SSH": https://help.ubuntu.com/lts/serverguide/openssh-server.html\\

My IP address: \texttt{hostname -I}\\
List of logins to the server: \texttt{sudo less /var/log/auth.log}\\

\vfill \null
\columnbreak
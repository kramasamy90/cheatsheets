\section{Networking}

\begin{tabularx}{\linewidth}{lX}
\hline
\texttt{wget \textit{url}} & Download file from http or ftp.\\
\texttt{-- accept, -A "..."} & Only download files matching this criteria. Eg "*.fastq"\\
\texttt{-- reject, -R} & Similar to above\\
\texttt{--no-directory, -nd} & Don't download directory structure.Only files.\\
\texttt{--recursive, -r} & \\
\texttt{--no-parent, -np} & Don't move above parent directory. This is important to avoid downloading unnecessary data.\\
\texttt{-O} & Output filename.\\
\texttt{-e robots=off} & To not want wget to follow '\texttt{robot.txt}'.\\
& See: \href{https://stackoverflow.com/a/11124664}{This answer} \\
\hline
\end{tabularx}

\textcolor{red}{Other options: --limit-rate, --user=user, --ask-password}\\


\begin{tabularx}{\linewidth}{lX}
\hline
\texttt{curl \textit{url} > file} & Redirect output to file.\\
\texttt{curl -O <file>} & download to file.\\
\texttt{-L,--location} & Download ultimate page and not the redirect page.\\
\hline
\end{tabularx}
Curl can also download form SFTP and SCP. Also checkout RCurl and pycurl.

\begin{tabularx}{\linewidth}{lX}
\hline
\texttt{shasum} & Calculate checksum using SHA-1. Can be used to find checksum of many files and store the result in a text file. Eg: \texttt{shasum *.fa > chksm.sha}\\
\texttt{-c} & Validate the files. Eg: \texttt{shasum -c chksm.sha}.\\
\texttt{sum} & Checksum program used by Ensemble.\\
\texttt{diff -u} & Outputs a diff file that shows difference between two files. Eg: \texttt{diff -u file1 file2}\\
\hline
\end{tabularx}

\vfill \null
\columnbreak

\subsection{SSH}
\begin{itemize}
\item \textbf{Usage}\\
\texttt{\$ ssh host}\\
\texttt{\$ password:}\\

\item \textbf{Examples of host}\\
\texttt{192.162.82.120}\\
\texttt{bioclust.myuniversity.edu}\\
\texttt{darwin@192.162.82.120}\\
\texttt{darwin@bio.univ.edu}\\

\item \textbf{Options} \\

\texttt{-v} verbose. Verbosity can be increased by: \texttt{-vv} or \texttt{-vvv}. \\
\texttt{-p} port. Eg: \texttt{ssh -p 5043 cdarwin@bio.univ.edu}\\
\qquad Default port is \texttt{22}

 
\item \textbf{Using alias:} To use alias create the file \texttt{\~{}/.ssh/config} and store server as info as below. 
\texttt{Host bio\_serv}\\
\qquad \texttt{HostName 190.512.171.29}\\
\qquad \texttt{User cdarwin}\\
\qquad \texttt{Port 50434}\\
Also applies for \texttt{Rsync and scp}\\

\item \textbf{SSH keys}: SSH key to connect without password. Eg:\\
\texttt{\$ ssh-keygen -b 2048}\\
This command request the following:\\
\begin{itemize}
\item File to save the key. By default this is:\\
\texttt{/Users/username/.ssh/id\_rsa}
NOTE: This file is the private key.\\
\item Passphrase:\\
Not necessary but good to use.\\
\end{itemize}
Private key: \texttt{~/.ssh/id\_rsa}\\
Public key: \texttt{~/.ssh/id\_rsa.pub}\\

\texttt{\$ chmod 400 id\_rsa \# restrict access to private key}\\
\texttt{\$ ssh-add} \\

\end{itemize}


\subsection{Setting up a server}
Use "Open SSH": https://help.ubuntu.com/lts/serverguide/openssh-server.html\\

My IP address: \texttt{hostname -I}\\
List of logins to the server: \texttt{sudo less /var/log/auth.log}\\

\vfill \null
\columnbreak
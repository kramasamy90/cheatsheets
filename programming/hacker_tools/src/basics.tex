\section{Navigation, Files and Directories.}

%\textbf{Files and Directories}
\begin{tabularx}{\linewidth}{lX}
    % Navigate and explore.
    \texttt{cd} & Change directory.\\
    \texttt{pwd} & Print working directory\\
    \texttt{ls} & Options: \texttt{-Ralh} \\
    \texttt{tree} & List in tree form. eg: \texttt{tree \textit{dir}}\\
    \hline

    % Manipulate files and directories.
    \texttt{touch} & Creates text file. \\
    \texttt{mkdir} & Make directory\\
    \texttt{mkdir -p} & Make directory and necessary parent dir.\\
    \texttt{cp} & Copy files.\\
    \texttt{mv} & To move files and rename files.\\
    \texttt{rm} & Remove files permanently. \\
    \texttt{rm -i} & Remove files interactively. \\
    \texttt{rm -r, rm -R} & Remove files recursively. Use to delete folders.\\
    \texttt{rm -f} & Force delete.\\
    \texttt{basename} & Removes folder name from path and optionally  suffix.\\
    \texttt{-s} & Remove suffix. eg: \texttt{basename -s .fastq $<$\textit{path}$>$}\\
    \hline

    % Common aliases.
    \texttt{\~} & Home directory, aka \texttt{\$HOME}.\\
    \texttt{./ , ../} & Relative paths to current and parent dir.\\
    \texttt{/dev/null} & Fake file, black box.\\ 
    \hline

    %
    % File permissions.
    \texttt{chmod 777} & r-4,w-2,x-1. User, group,all.  \\
    \texttt{chmod xyz} & Eg \texttt{chmod u+w}.\\
    & x = \texttt{u} : user, \texttt{g} : group, \texttt{a} : all.\\
    & y = \texttt{+} : add, \texttt{-} : remove.\\
    & z = \texttt{r} : read, \texttt{w} : write, \texttt{x} : execute.\\
    \hline

    % Disk usage. 
    \texttt{du -h \textit{dir}} & Gives size of all directories in \texttt{\textit{dir}}\\
    \texttt{du -sh \textit{dir}} & Gives size of \texttt{\textit{dir}}.\\
    \texttt{df -h} & Gives information about disk usage.\\
    \hline
    
    % Pattern matching.
    \texttt{?,*, [A-Z]} & Wild cards.\\
    \texttt{\{\}} & Expands combinatorially.\\ 
    &  Eg: \texttt{\$ mkdir mm10-\{chr1,chr2,chr3\}}\\
    \$() & Eg: \texttt{echo "...\$(...)..."}\\
    & Eg: \texttt{mkdir results-\$(date +\%F)}\\
    \texttt{alias x ="..."} & Store new commands. But in shell startup file eg \texttt{\~{}/.profile} or \texttt{\~{}/.bashrc} and it is temporary.\\
    & Eg: \texttt{\$ today = "date + \%F"}. \\
    \hline
    % Process control & program execution.
    \texttt{echo \$?} & Exit status,=0 when a program exits without an error.\\
    \keys{ctrl} + \keys{c} & Kill a running job.\\
    \keys{ctrl} + \keys{z} & Pause a running job.\\
    \texttt{nohup} & Run a program without interruption. The program continues to run even after the user signs out. Useful to run a program for long time on a remote machine. Use run in the background using \texttt{\&} option.\\
    \texttt{\&} & Run in background. eg: \texttt{nohup prog1 \&}\\

\end{tabularx}

\vfill\null
\columnbreak 

\begin{tabularx}{\linewidth}{lX}
    \texttt{top} & Lists all ongoing processes. Better than \texttt{jobs}.\\
    \texttt{htop} & User friendly tool to view running processes and resource utilization.\\
    \texttt{jobs} & List all jobs. Use id in [] to bg,fg,kill. \\
    \texttt{fg,bg,kill} & Bring job to foreground, background, kill a job. Pause a job before bg.  \\

    \texttt{Cmd1 ; Cmd2} & Run Cmd2 irrespective of exit status of Cmd1.\\
    \texttt{Cmd1 || Cmd2} & Execute \texttt{Prog2} only if \texttt{Prog1} has failed (non-zero exit status).\\
    \texttt{Cmd1 \&\& Cmd2} & Execute \texttt{Prog2} only if \texttt{Prog1} has succeded (zero exit status).\\
    \hline
\end{tabularx}

\textcolor{red}{TODO: chown, chgrp.}\\
\textcolor{red}{File management: rsync, bzip, tar}\\
\textcolor{red}{hexdump, checksums, diff (in text processing?).}\\
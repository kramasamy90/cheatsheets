\section{Shell scripting}
\subsubsection{Modifying PATH}
Add a directory to path: Append one of the following files.\\
\texttt{$\sim$/.profile} or \texttt{$\sim$/.bash\_profile}\\
with the following line:\\
\texttt{PATH=\$PATH:\textit{<directory>}}\\
Eg: \texttt{PATH=\$PATH:\$HOME/scripts}\\

%\textcolor{blue}{NOTE: the above line could used in a terminal and the modification of PATH variable is valid only during the session, alternatively when it is used within a script the modification is valid only within the script and for example not in the terminal where the script is run.}\\


\subsubsection{Header}

\begin{tabularx}{\linewidth}{lX}
\texttt{\#!/bin/bash} & Shebang\\
\texttt{set -e} & Terminates script if there is non-zero exit status.\\
\texttt{set -o pipefail} & If a program in the pipe fails the entire pipe returns non-zero exit status.\\
\texttt{set -u} & Terminates for undefined variables.\\
\end{tabularx}

\subsubsection{Variables}
\begin{tabularx}{\linewidth}{lX}
\texttt{sample="CNTRL"} & Assignment, no space around "\texttt{=}"\\
\texttt{echo \$sample} & \\
\texttt{echo \$\{sample\}\_aln} & Use curly braces while concatenating a variable with additional text.\\
\texttt{mkdir "\$\{sample\}\_aln}" & Quoting variables prevents commands from interpreting spaces and special variables.\\
\texttt{echo \$\{\#sample\}} & Length of the variable \texttt{sample}
\end{tabularx}

\subsubsection{Command-line arguments}
\begin{tabularx}{\linewidth}{lX}
\texttt{\$0} & Script name\\
\texttt{\$1} & First argument\\
\texttt{\$n} & n$^{th}$ argument.\\
\texttt{\$\#} & Number of arguments not including \texttt{\$0}.\\
\end{tabularx}

\textbf{Example:}\\
\begin{mdframed}
\texttt{\#!/bin/bash}\\
\texttt{echo "script name: \$0"}\\
\texttt{echo "first arg: \$1"}\\
\texttt{echo "second arg: \$2"}\\
\texttt{echo "There are \$\# input arguments"}
\end{mdframed}

\subsection{Conditionals}
\textbf{Format}
\begin{mdframed}
\texttt{if [ \textit{<conditon-statement>} ]}\\
\texttt{then}\\
\texttt{if-statements}\\
\texttt{elif}\\
\texttt{then}\\
\texttt{elif-statements}\\
\texttt{else}\\
\texttt{else-statements}\\
\texttt{fi}
\end{mdframed}

\columnbreak
\vfill \null

\textbf{Example:}\\
\begin{mdframed}
\texttt{if [ \$\# -lt 3 ]}\\
\texttt{then}\\
\texttt{echo "There are less than 3 arguments"}\\
\texttt{fi}
\end{mdframed}

\textcolor{red}{In bash 0 is true/success, anything else is false/failure}

\textbf{String and integer comparison}
\begin{tabularx}{\linewidth}{lX}
\texttt{-z str} & \texttt{str}  is null string.\\
\texttt{str1 == str2} & \texttt{str1} and \texttt{str2} are identical.\\
\texttt{str1 != str2} & \\
\texttt{int1 -eq int2} & \texttt{int1} and \texttt{int2} are equal.\\
\texttt{int1 -ne int2} & \\
\texttt{int1 -lt int2} & \\
\texttt{int1 -gt int2} & \\
\texttt{int1 -le int2} & \\
\texttt{int1 -ge int2} & \\
\texttt{-o} & Logical OR.\\
\texttt{-a} & Logical AND.\\
\end{tabularx}

\texttt{if} conditional can also be used to depend on exit status. Eg:\\

\begin{mdframed}[nobreak=true]
\texttt{if grep "pattern" file1.txt > /dev/null \&\& grep "pattern" file2.txt > /dev/null/}\\
\texttt{then}\\
\texttt{    echo "found pattern in file1.txt and file2.txt"}\\
\texttt{fi}
\end{mdframed}

\begin{mdframed}
\texttt{if ! grep "pattern" file1.txt > /dev/null}\\
\texttt{then}\\
\texttt{    echo "pattern not found in file1.txt"}\\
\texttt{fi}
\end{mdframed}

\subsubsection{Testing files and dirs}
\textbf{List of test expressions.}\\

\begin{tabularx}{\linewidth}{lX}
\texttt{-d dir} & \texttt{dir} is a directory\\
\texttt{-f file} & \texttt{file} is a file.\\
\texttt{-e file} & \texttt{file} exists.\\
\texttt{-h lind} & \texttt{link} is a link.\\
\texttt{-r file} & \texttt{file} is readable.\\
\texttt{- w file} & \texttt{file} is writable.\\
\texttt{-x file} & \texttt{file} is executable.\\
\end{tabularx}

Example\\

\begin{mdframed}
\texttt{test -d dir ; echo \$? }\\
\null\\
\texttt{test -d dir1 -o -d dir2; echo \$?}

\end{mdframed}





Exit status would be \texttt{0} if the directory dir exists.\\


\textbf{Example:}\\

\begin{mdframed}
\texttt{if ! test -d \$1}\\
\texttt{then}\\
\texttt{mkdir \$1}\\
\texttt{fi}
\end{mdframed}

Above script is equivalent to the following.\\

\begin{mdframed}
\texttt{if [ ! -d \$1 ]}\\
\texttt{then}\\
\texttt{mkdir \$1}\\
\texttt{fi}
\end{mdframed}

\null

\subsection{Arrays and For loop}

\subsubsection{Manual creation}
\begin{mdframed}

\texttt{\$ sample\_names=(zmaysA zmaysB zmaysC)}\\
\texttt{\$ echo \$\{sample\_names[0]\}}\\
\texttt{zmaysA}\\
\texttt{\$ echo \$\{sample\_names[@]\}}\\
\texttt{zmaysA zmaysB zmaysC}\\
\texttt{\$ echo \$\{\#sample\_names[@]\}}\\
\texttt{3}\\
\texttt{\$ echo \$\{!sample\_names[@]\}}\\
\texttt{0 1 2}
\end{mdframed}

\subsubsection{Array creation using command substitution}
\begin{mdframed}
\texttt{samples=(\$(cut -f3 samples.tsv))}\\
\texttt{file\_names=(\$(ls))}
\end{mdframed}


\subsubsection{Array of number sequence}

\begin{mdframed}
\texttt{seq 0 0.1 1 \# seq start step end}\\
\texttt{s=(\$(seq 0 0.1 1))}
\end{mdframed}


\begin{tabularx}{\linewidth}{lX}
\texttt{\$\{arr[i]\}} & (i-1)$^{th}$ element of \texttt{array}.\\
\texttt{\$\{arr[@]\}} & All the elements of \texttt{arr}.\\
\texttt{\$\{\#sample\_names[@]\}} & Length of \texttt{arr}.\\
\texttt{\$\{!sample\_names[@]\}} & Returns an array containing the index of elements in \texttt{arr}.\\
\end{tabularx}

\subsection{For loop}


\begin{mdframed}
\texttt{for name in \$\{file\_names[@]\}}\\
\texttt{do}\\
\quad\texttt{process.sh \$name}\\
\texttt{done}\\
\null\\
\texttt{for name in \$\{file\_names[@]\}; do}\\
\quad\texttt{process.sh \$name}\\
\texttt{done}\\
\null\\
\texttt{for name in \$\{file\_names[@]\}; do; process.sh \$name; done }\\
\texttt{for i in \$(seq \textit{start} \textit{step\_size} \textit{end});}\\
\texttt{do}\\
\texttt{process.sh \$i}\\
\texttt{done}
\end{mdframed}




\subsection{Find, exec and xargs}

\begin{tabularx}{\linewidth}{lX}
    %
    \texttt{expr -and expr} & Logical AND.\\
    \texttt{expr -or expr} & Logial OR.\\
    \texttt{-not expr} & Logial NOT. Alternate: \texttt{"!" expr}\\
    \texttt{(expr)} & Group a set of expressions.\\
    %
    \texttt{-exec} & Example: \\
    & \texttt{find . -name *.c -exec <prog1> \{\} \textbackslash ;}.\\
    & Execute \texttt{<prog1>} on all the found files. \texttt{\{\}} represents the found files. \\
    & Mind the space between \texttt{\{\}} and \texttt{\textbackslash;}\\
\hline

\end{tabularx}




\subsection{Arithmetics}
\subsubsection{let}
Examples using \texttt{let}:\\

\begin{mdframed}
\texttt{let x=1} \#No space within expression\\
\texttt{let x=x*2}\\
\texttt{let x++}\\
\texttt{let "x = x + 1"} \# Space OK within quotation.
\end{mdframed}

Examples using \texttt{expr}:
\begin{mdframed}
\texttt{expr 2 + 3} \# Space is required for \texttt{expr}\\
\texttt{a=\$(expr 2 + 3)}\\
\texttt{expr \$x + 1}
\end{mdframed}

\texttt{expr} is simillar to \texttt{let}, but only evaluate and not assign value to a variable.\\

\textbf{Arithmetic operations:}

\begin{tabularx}{\linewidth}{lX}
+, -, /, \% & \\
* & Multiplication operator for \texttt{let}\\
/* & Multiplication operator for \texttt{expr}\\
\texttt{var++} & increment var by 1 used only in \texttt{let}\\
\texttt{var--} & increment var by 1 used only in \texttt{let}
\end{tabularx}

%%%%%%%%%%%%%%%%%%%%%%%%%%%%%%%%%%%%%%%%%%%%%%%%%%%%%%%%%%%%%%%%%%%%%%%%%%%%%

\vfill\null
\pagebreak


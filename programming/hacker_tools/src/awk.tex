\section{Awk}

Format: \texttt{awk pattern \{action\} input1.txt input2.txt} \\ \texttt{awk -f file.awk input.txt}.\\ Record = row. Column = fields. \\

\begin{tabularx}{\linewidth}{lX}

\texttt{-F} & Input field separator. Eg: \texttt{awk -F"," input.txt}. Defaule field separator = tab.\\
\texttt{-f} & Take input from file. Eg: awk -f file.awk input.txt.\\
\texttt{(...) \&\& (...)} &  Use logical operators. See below.\\
\texttt{\$n ~ /.../} & Use regular expression between slashes.\\
\texttt{/.../,/.../} & Specify range. Works only with regex (with double slash) .\\
\texttt{BEGIN\{...\}} & Eg: \texttt{awk 'BEGIN\{...\} ... \{...\} END\{...\}'}\\
\texttt{END\{...\}} & \\
\hline
\end{tabularx}
\textbf{Awk operations:} \texttt{+,-,*,/,\%,\^{}}. \\
\texttt{a ... b}. Replace "\texttt{...}" : \texttt{==,!=,<,>,<=,>=,\~{},!\~{},\&\&,||,!a}\\
\textbf{Field separators:} FS,RS,OFS, ORS.\\
\textbf{Awk variables:} NF, NR (Record number accumulates between files.), FNR(Resets record number after every files.).\\

\begin{tabularx}{\linewidth}{lX}
\hline
\end{tabularx}

\textbf{Example awk script file}\\
\texttt{awk -f script.awk plasmids.tsv}\\
\texttt{BEGIN\{FS="\textbackslash t";OFS="\textbackslash t";x=0\}}\\
\texttt{/[Cc]re/\{}\\
\texttt{x+=1;}\\
\texttt{print x,\$1,\$2\}}\\
\texttt{END\{print "There are " x "plasmids with Cre"\}}

\begin{tabularx}{\linewidth}{lX}
\hline
\end{tabularx}

\textcolor{red}{Checkout BioAwk}.\\
\textcolor{red}{Checkout control flow}.\\
\begin{tabularx}{\linewidth}{lX}
\hline
\end{tabularx}

\section{Sed}

\texttt{sed 's/\textit{target}/\textit{replacement}/\textit{flag}'}

\begin{tabularx}{\linewidth}{lX}
\texttt{-e} & to Chain commands.Eg: \texttt{sed -e 's/:/\t/' -e 's/-/\t/'}.\\
\texttt{-E} & Use extended POSIX.\\
\texttt{g} & Global flag. Usually sed replaces only the first occurrence  in a sentence. Use global flag to replace all occurences.\\
\texttt{i} & To make the search case insensitive.\\
\end{tabularx}

%\hspace{10pt}\\
%\hspace{10pt}\\
%\hspace{10pt}\\

%\pagebreak

\section{Regular Expression}

\textbf{Single character meta characters}
\begin{tabularx}{\linewidth}{lX}

\texttt{.} & Match any single character.\\
\texttt{[ ]} & Match any single character between \texttt{[]}. Eg: \texttt{[at]} match "a" or "t".\\
\texttt{[\^{}]} & Match any single charcter except on between \texttt{[]}.\\
\texttt{[0-9]} & Any number between 0 and 9. Eg: \texttt{0-3a-cz]} equals \texttt{[123abcz]}.\\
\texttt{(...)} & Grouping. eg: \texttt{(AT)+} or \texttt{(GLY ) \{2,\}}.\\
\hline

\end{tabularx}


\textbf{Quantifiers}
\begin{tabularx}{\linewidth}{lX}
\texttt{?} & Match preceding character zero or one time. \\
\texttt{*} & Match zero or more time.\\
\texttt{+} & Match one or more time.\\
\texttt{\{n\}} & Match n times.\\
\texttt{\{n,\}} & Match atleast n times.\\
\texttt{\{a,b\}} & Match atleast a times, atmost b times.\\

\hline

\end{tabularx}

\textbf{Anchors}

\begin{tabularx}{\linewidth}{lX}

\texttt{\^{}} & Match the start of a line.\\
\texttt{\$} & Match end of a line.\\
\texttt{\textbackslash $<$} & Match beginning of word.\\
\texttt{\textbackslash $>$} & Match at the end of word.\\
\texttt{\textbackslash b} & Match either beginning or end of word.\\
\texttt{\textbackslash B} & Match any character not at the beginning or end. \\

\hline

\end{tabularx}

\textbf{Character class}

[:alnum:], [:digit:], [:alpha:], [:upper:], [:lower:], [:blank:], [:space:], [:punct:] and [:print:].\\

Use backslash as escape character. \\
\begin{tabularx}{\linewidth}{lX}
\texttt{\textbackslash s} & white space character. What it includes depends on the flavour of regex.\\
\texttt{\textbackslash d} & Add digits.\\
\texttt{\textbackslash w} & Word character, matches \texttt{[A-Za-z0-9\_]}\\
\end{tabularx}
\texttt{|} as OR logical operator: \texttt{(GLY|GLN)}.
\texttt{"one and|or two"} is equal to \texttt{"(one and)|(or two)"}.\\
\texttt{"one (and|or) two"} is "one and two" or "one or two".\\

\textbf{Back references:} () : Memorizes the match for regular expression within parenthesis. Use \texttt{\textbackslash n} to recall nth match.

\begin{tabularx}{\linewidth}{lX}
\hline
\end{tabularx}

\vfill\null
\columnbreak

\section{Vim}

\textbf{Motion}
Usage: \texttt{<num> <motion>}
\begin{tabularx}{\linewidth}{lX}
\texttt{h l} & One character left or right.\\
\texttt{j k} & One line up or down.\\
\texttt{w b} & One word forward or backwarks.\\
\texttt{e} & Simillar to w but keeps the cursor at the end of the word.\\
\texttt{0} & Cursor to the begining of the sentence.\\
\texttt{\$} & Moves cursor to the end of the sentences.\\
\texttt{G} & End of the file.\\
\texttt{gg} & First line.\\
\texttt{H} & Top of screen.\\
\texttt{M} & Middle of screen.\\
\texttt{L} & Botom of screen.\\
\texttt{<num>G} & Go to line \texttt{<num>}.\\
\keys{\texttt{Ctrl}} + \texttt{f} & One screen forward.\\
\keys{\texttt{Ctrl}} + \texttt{b} & One screen backward.\\
\keys{\texttt{Ctrl}} + \texttt{G} & View position in the file.\\
\keys{\texttt{Ctrl}} + \texttt{O} & Go to where you came from .\\
\keys{\texttt{Ctrl}} + \texttt{I} & Opposite of \keys{\texttt{Ctrl}} + \keys{\texttt{O}}\\
\texttt{\%} & Go to the corresponding opening or closing parenthesis.\\
\hline\\

\end{tabularx}

\textbf{Operators}

\begin{tabularx}{\linewidth}{lX}
\texttt{i} & INSERT mode\\
\texttt{a} & append, goes to insert mode\\
\texttt{a} & append from the end of the line.\\
\texttt{v} & visual selection, selection is stored in clipboard\\
\texttt{o} & open a line below\\
\texttt{O} & open a line above\\
\keys{\texttt{Esc}} & Go to command mode\\
\texttt{d} & delete and also cut, $\equiv$ \keys{\texttt{Ctrl}} + \keys{\texttt{X}}\\
\texttt{dd} & delete whole sentence\\
\texttt{x} & delete character under the cursor\\
\texttt{r} & replace the character under the cursor\\
\texttt{R} & replace until \keys{\texttt{Esc}} \\
\texttt{c} & change: works equivalent to \texttt{d} followed by \texttt{i}\\
\texttt{y} & yank, copy\\
\texttt{p} & paste\\
\texttt{u} & undo most recent edit\\
\texttt{U} & undo all the changes in the line\\
\keys{\texttt{Ctrl}} + \keys{\texttt{R}} & Redo\\
\hline

\end{tabularx}

\textbf{Copy, paste, bookmark}

\begin{tabularx}{\linewidth}{lX}

\texttt{:xmy} & Move line \texttt{x} below line \texttt{y}.\\
\texttt{:x,ymz} & Moves lines between and including \texttt{x} and \texttt{y} below line \texttt{z}.\\
\texttt{:xty} & Copy line \texttt{x} below line \texttt{y}.\\
\texttt{:x,ytz} & Copy lines between and including \texttt{x} and \texttt{y} below line \texttt{z}.\\
\texttt{ma} & Set bookmark at current line. \texttt{a} $\in$ [a-z].\\
\texttt{'a} & Jump to bookmark \texttt{a}.\\
\texttt{:'a,'bco'c} & Copy lines between and including bookmarks \texttt{a} and \texttt{b} below bookmark c.\\
\texttt{:'a,'bco'z} & Copy lines between and including bookmarks \texttt{a} and \texttt{b} below line z.\\
\hline\\
\end{tabularx}

\textbf{Search and replace}\\

\begin{tabularx}{\linewidth}{lX}
:/REGEX & Find regular expression.\\
n & next search target\\
N & Previous search target\\
:s/target/replace & Simillar to sed. Replaces target only in the current sentence and only once.\\
:s/target/replace/g & Replaces at all instance in the current sentence.\\
:\%s/target/replace/g & Replaces through the entire file.\\
:\%s/target/replace/gc & Ask for confirmation at each instance.\\

\end{tabularx}

\begin{tabularx}{\linewidth}{lX}
\hline
\end{tabularx}

\textbf{Save, write and Exit}

\begin{tabularx}{\linewidth}{lX}
\texttt{:q} & quit\\
\texttt{:q!} & quit without saving\\
\texttt{:w} & save the current file\\
\texttt{:wq} or \texttt{:x} & save and quite \\
\texttt{:w \textit{file}} & write to \textit{file}.\\
\texttt{:xyw \textit{file}} & write lines between and including lines \texttt{x} and \texttt{y} to \textit{file}.\\
\texttt{:!} & Execute shell command. Eg: \texttt{:!pwd}\\
\hline\\

\end{tabularx}

\textbf{\texttt{:set}}\\
Usage: \texttt{:set \textit{option}}. Eg: \texttt{:set ic}
\begin{tabularx}{\linewidth}{lX}
ic & Case-insensititve search\\
hls & Highlight search\\
number & Show line number \\
\end{tabularx}
To turnoff the option use no. Eg \texttt{:set noic} to turnoff \texttt{ic}\\
\begin{tabularx}{\linewidth}{lX}
\hline\
\end{tabularx}
\textbf{Etc}

\keys{Ctrl} + \keys{D} for command completion.\\
\keys{Tab} for filename completion.\\
For further setting: \texttt{$\sim$/.vimrc} \\
\textbf{Help}:\\
\keys{F1}\\
\texttt{:help}\\
\texttt{:help w}\\
\texttt{:help user-manual}\\

\textbf{Default settings:}
Set default settings in \texttt{~/.vimrc} \\
Create this file if it does not exist. \\
Example \texttt{.vimrc} file:\\

\begin{mdframed}
\texttt{syntax on}\\
\texttt{colorscheme desert}\\
\texttt{set number}\\
\texttt{set hls}

\end{mdframed}

\vfill\null

\columnbreak

\section{Shell scripting}
\subsubsection{Modifying PATH}
Add a directory to path: Append one of the following files.\\
\texttt{$\sim$/.profile} or \texttt{$\sim$/.bash\_profile}\\
with the following line:\\
\texttt{PATH=\$PATH:\textit{<directory>}}\\
Eg: \texttt{PATH=\$PATH:\$HOME/scripts}\\

%\textcolor{blue}{NOTE: the above line could used in a terminal and the modification of PATH variable is valid only during the session, alternatively when it is used within a script the modification is valid only within the script and for example not in the terminal where the script is run.}\\


\subsubsection{Header}

\begin{tabularx}{\linewidth}{lX}
\texttt{\#!/bin/bash} & Shebang\\
\texttt{set -e} & Terminates script if there is non-zero exit status.\\
\texttt{set -o pipefail} & If a program in the pipe fails the entire pipe returns non-zero exit status.\\
\texttt{set -u} & Terminates for undefined variables.\\
\end{tabularx}

\subsubsection{Variables}
\begin{tabularx}{\linewidth}{lX}
\texttt{sample="CNTRL"} & Assignment, no space around "\texttt{=}"\\
\texttt{echo \$sample} & \\
\texttt{echo \$\{sample\}\_aln} & Use curly braces while concatenating a variable with additional text.\\
\texttt{mkdir "\$\{sample\}\_aln}" & Quoting variables prevents commands from interpreting spaces and special variables.\\
\texttt{echo \$\{\#sample\}} & Length of the variable \texttt{sample}
\end{tabularx}

\subsubsection{Command-line arguments}
\begin{tabularx}{\linewidth}{lX}
\texttt{\$0} & Script name\\
\texttt{\$1} & First argument\\
\texttt{\$n} & n$^{th}$ argument.\\
\texttt{\$\#} & Number of arguments not including \texttt{\$0}.\\
\end{tabularx}

\textbf{Example:}\\
\begin{mdframed}
\texttt{\#!/bin/bash}\\
\texttt{echo "script name: \$0"}\\
\texttt{echo "first arg: \$1"}\\
\texttt{echo "second arg: \$2"}\\
\texttt{echo "There are \$\# input arguments"}
\end{mdframed}

\subsection{Conditionals}
\textbf{Format}
\begin{mdframed}
\texttt{if [ \textit{<conditon-statement>} ]}\\
\texttt{then}\\
\texttt{if-statements}\\
\texttt{elif}\\
\texttt{then}\\
\texttt{elif-statements}\\
\texttt{else}\\
\texttt{else-statements}\\
\texttt{fi}
\end{mdframed}

\columnbreak
\vfill \null

\textbf{Example:}\\
\begin{mdframed}
\texttt{if [ \$\# -lt 3 ]}\\
\texttt{then}\\
\texttt{echo "There are less than 3 arguments"}\\
\texttt{fi}
\end{mdframed}

\textcolor{red}{In bash 0 is true/success, anything else is false/failure}

\textbf{String and integer comparison}
\begin{tabularx}{\linewidth}{lX}
\texttt{-z str} & \texttt{str}  is null string.\\
\texttt{str1 == str2} & \texttt{str1} and \texttt{str2} are identical.\\
\texttt{str1 != str2} & \\
\texttt{int1 -eq int2} & \texttt{int1} and \texttt{int2} are equal.\\
\texttt{int1 -ne int2} & \\
\texttt{int1 -lt int2} & \\
\texttt{int1 -gt int2} & \\
\texttt{int1 -le int2} & \\
\texttt{int1 -ge int2} & \\
\texttt{-o} & Logical OR.\\
\texttt{-a} & Logical AND.\\
\end{tabularx}

\texttt{if} conditional can also be used to depend on exit status. Eg:\\

\begin{mdframed}[nobreak=true]
\texttt{if grep "pattern" file1.txt > /dev/null \&\& grep "pattern" file2.txt > /dev/null/}\\
\texttt{then}\\
\texttt{    echo "found pattern in file1.txt and file2.txt"}\\
\texttt{fi}
\end{mdframed}

\begin{mdframed}
\texttt{if ! grep "pattern" file1.txt > /dev/null}\\
\texttt{then}\\
\texttt{    echo "pattern not found in file1.txt"}\\
\texttt{fi}
\end{mdframed}

\subsubsection{Testing files and dirs}
\textbf{List of test expressions.}\\

\begin{tabularx}{\linewidth}{lX}
\texttt{-d dir} & \texttt{dir} is a directory\\
\texttt{-f file} & \texttt{file} is a file.\\
\texttt{-e file} & \texttt{file} exists.\\
\texttt{-h lind} & \texttt{link} is a link.\\
\texttt{-r file} & \texttt{file} is readable.\\
\texttt{- w file} & \texttt{file} is writable.\\
\texttt{-x file} & \texttt{file} is executable.\\
\end{tabularx}

Example\\

\begin{mdframed}
\texttt{test -d dir ; echo \$? }\\
\null\\
\texttt{test -d dir1 -o -d dir2; echo \$?}

\end{mdframed}





Exit status would be \texttt{0} if the directory dir exists.\\


\textbf{Example:}\\

\begin{mdframed}
\texttt{if ! test -d \$1}\\
\texttt{then}\\
\texttt{mkdir \$1}\\
\texttt{fi}
\end{mdframed}

Above script is equivalent to the following.\\

\begin{mdframed}
\texttt{if [ ! -d \$1 ]}\\
\texttt{then}\\
\texttt{mkdir \$1}\\
\texttt{fi}
\end{mdframed}

\null

\subsection{Arrays and For loop}

\subsubsection{Manual creation}
\begin{mdframed}

\texttt{\$ sample\_names=(zmaysA zmaysB zmaysC)}\\
\texttt{\$ echo \$\{sample\_names[0]\}}\\
\texttt{zmaysA}\\
\texttt{\$ echo \$\{sample\_names[@]\}}\\
\texttt{zmaysA zmaysB zmaysC}\\
\texttt{\$ echo \$\{\#sample\_names[@]\}}\\
\texttt{3}\\
\texttt{\$ echo \$\{!sample\_names[@]\}}\\
\texttt{0 1 2}
\end{mdframed}

\subsubsection{Array creation using command substitution}
\begin{mdframed}
\texttt{samples=(\$(cut -f3 samples.tsv))}\\
\texttt{file\_names=(\$(ls))}
\end{mdframed}


\subsubsection{Array of number sequence}

\begin{mdframed}
\texttt{seq 0 0.1 1 \# seq start step end}\\
\texttt{s=(\$(seq 0 0.1 1))}
\end{mdframed}


\begin{tabularx}{\linewidth}{lX}
\texttt{\$\{arr[i]\}} & (i-1)$^{th}$ element of \texttt{array}.\\
\texttt{\$\{arr[@]\}} & All the elements of \texttt{arr}.\\
\texttt{\$\{\#sample\_names[@]\}} & Length of \texttt{arr}.\\
\texttt{\$\{!sample\_names[@]\}} & Returns an array containing the index of elements in \texttt{arr}.\\
\end{tabularx}

\subsection{For loop}


\begin{mdframed}
\texttt{for name in \$\{file\_names[@]\}}\\
\texttt{do}\\
\quad\texttt{process.sh \$name}\\
\texttt{done}\\
\null\\
\texttt{for name in \$\{file\_names[@]\}; do}\\
\quad\texttt{process.sh \$name}\\
\texttt{done}\\
\null\\
\texttt{for name in \$\{file\_names[@]\}; do; process.sh \$name; done }\\
\texttt{for i in \$(seq \textit{start} \textit{step\_size} \textit{end});}\\
\texttt{do}\\
\texttt{process.sh \$i}\\
\texttt{done}
\end{mdframed}




\subsection{Find, exec and xargs}

\begin{tabularx}{\linewidth}{lX}
\texttt{find} & Usage: \texttt{find \textit{<folder>} -name "\textit{<pattern>}"}.\\
& Eg: \texttt{find . -name foo.sh}.\\
\texttt{-name <pattern>} & Find \texttt{<pattern>} using same special characters as bash (\texttt{*,?, [...] })\\
\texttt{-iname} & Identical to \texttt{-name} but case-insensitive.\\
\texttt{-empty} & Matches emtpy files and folders.\\
\texttt{-type <x>} & Matches types \texttt{x} (\texttt{f} - file, \texttt{d} - directory, \texttt{l} - links).\\
\texttt{-size <size>} & Matches \texttt{<size>}.\\
&  Eg: \texttt{+50M} ; Files larger than 50 MB\\
& Eg: \texttt{-50M}; Files smaller than 50 MB\\
\end{tabularx}

\begin{tabularx}{\linewidth}{lX}

\texttt{-regex} & Match regular expression. Use \texttt{-E} for extended POSIX.\\
\texttt{-iregex} & Case-insensitive.\\
\texttt{-print0} & separate results with null-byte and not new line. 
\textcolor{red}{Explain!}\\
\texttt{expr -and expr} & Logical AND.\\
\texttt{expr -or expr} & Logial OR.\\
\texttt{-not expr} & Logial NOT. Alternate: \texttt{"!" expr}\\
\texttt{(expr)} & Group a set of expressions.\\
\texttt{-exec} & Example: \\
 & \texttt{find . -name *.c -exec <prog1> \{\} \textbackslash ;}.\\
 & Execute \texttt{<prog1>} on all the found files. \texttt{\{\}} represents the found files. \\
 & Mind the space between \texttt{\{\}} and \texttt{\textbackslash;}\\
 
\hline

\end{tabularx}




\subsection{Arithmetics}
\subsubsection{let}
Examples using \texttt{let}:\\

\begin{mdframed}
\texttt{let x=1} \#No space within expression\\
\texttt{let x=x*2}\\
\texttt{let x++}\\
\texttt{let "x = x + 1"} \# Space OK within quotation.
\end{mdframed}

Examples using \texttt{expr}:
\begin{mdframed}
\texttt{expr 2 + 3} \# Space is required for \texttt{expr}\\
\texttt{a=\$(expr 2 + 3)}\\
\texttt{expr \$x + 1}
\end{mdframed}

\texttt{expr} is simillar to \texttt{let}, but only evaluate and not assign value to a variable.\\

\textbf{Arithmetic operations:}

\begin{tabularx}{\linewidth}{lX}
+, -, /, \% & \\
* & Multiplication operator for \texttt{let}\\
/* & Multiplication operator for \texttt{expr}\\
\texttt{var++} & increment var by 1 used only in \texttt{let}\\
\texttt{var--} & increment var by 1 used only in \texttt{let}
\end{tabularx}

%%%%%%%%%%%%%%%%%%%%%%%%%%%%%%%%%%%%%%%%%%%%%%%%%%%%%%%%%%%%%%%%%%%%%%%%%%%%%

\vfill\null
\pagebreak

\section{Git}


Setup git with the following commands:\\
\texttt{\$ git config --global user.name "Ramasamy Kandasamy"}\\
\texttt{\$ git config --global user.email ".....@gmail.com"}\\
Next command tells git to use color to indicate changes.\\
\texttt{\$ git config --global color.ui true}\\
To change default text editor:\\
\texttt{\$ git config --global core.editor gedit}\\
These commands create a .gitconfig file in home directory. Use \texttt{\$ cat \~{}/.gitconfig} to get current information.\\
\begin{tabularx}{\linewidth}{X}
\hline
\end{tabularx}


Git command structure: \texttt{git <subcommand>}
\begin{tabularx}{\linewidth}{lX}
\texttt{git init} & Initialize git repository in a directory.\\
\texttt{git clone} & To clone a git repository.\\ 
\end{tabularx}
Eg:\\
\texttt{\$ git clone https://github.com/user/sth.git}\\
\texttt{\$ git clone https://github.com/user/sth.git dir\_name}\\
\texttt{\$ git clone https://user@bitbucket.org/user/sth.git}\\
\begin{tabularx}{\linewidth}{X}
\hline 
\end{tabularx}


Git consists of untracked files, tracked file , files staged for commit, and files committed to the repository. \\
\begin{tabularx}{\linewidth}{lX}
\texttt{git status} & Gives three categories of files: untracked, tracked files that have been modified, files staged for commit.\\
\texttt{git add} & Start tracking a file or stage a file for commit.\\
\texttt{-f} & To stage a file not tracked, i.e. a file in \texttt{.gitignore}.\\
\texttt{git commit} & Commits all staged files to repository. \texttt{--amend}\\
\texttt{-a} & This options tells git to automatically stage all modified tracked files in this commit.\\
\texttt{-m "..."} & Message is mandatory. If there is no message, git opens text editor to input message. Default text editor can be specified in git-config.\\
\texttt{git diff} & Shows difference between current version and staged version. If there are no staged version, shows difference between last commit and current versions.\\
\texttt{-- staged} & To see difference between staged version and last commit.\\
\texttt{git reset} & Unstage a file. Without a file name all staged files get unstaged.\\
\texttt{git log} & List all commits, commit message SHA-1 checksum etc. Options: \texttt{--pretty=oneline}, \texttt{--abbrev-commit}, \texttt{--graph}, \texttt{--branches}, \texttt{-n2} : to view only latest two commits.\\
\texttt{git rm} & Use these commands to rename or delete files.\\
\texttt{git mv} & Using rm and mv will confuse git.\\
\texttt{.gitignore} & Used to avoid certain files, fastq files for example, from being listed in untracked section of \texttt{git status}.\\
 & Eg: \texttt{\$ echo "*.fa" >> .gitignore}.\\
\texttt{git ls-tree} & List contents of tree object.\\
 & Use to list all files in the latest commit.\\
 & Eg: \texttt{git ls-tree -r master --name-only}\\
\hline
\end{tabularx}

To add a remote repository.\\
\texttt{\$ git remote add origin git@github.com:username/project.git}\\
\texttt{\$ git remote add origin user@bitbucket....} \\

\begin{tabularx}{\linewidth}{lX}
\texttt{git remote -v} & Shows remote repository that connected to local repository. \\
\texttt{git remote rm} & Remove remote repository. Eg: \texttt{git remote rm origin}\\
\texttt{git push} & Use \texttt{git push origin master} to push main branch to origin (remote repository)\\
\texttt{git pull} & \texttt{git pull origin master}: simillar to above.\\
\hline\\
\end{tabularx}

\textbf{Resolving merge conflicts}: First \texttt{git pull} from remote repo. \texttt{git status} shows files with merge conflict. Open the file and resolve the conflict using guidlines provided. \\

\textcolor{red}{Unfinished: Github SSH}

\begin{tabularx}{\linewidth}{lX}
\texttt{git checkout} & Restores file from \texttt{HEAD}. To restore a file\\
\texttt{-- file} & from a specific commit. Use the commit SHA-1 ID. Eg \texttt{git checkout 08ccd3b -- README.md}\\
\texttt{git stash} & To temporarily store the changes and go back to \texttt{HEAD}.\\
 & \texttt{git stash pop} to restore changes stored in git stash.\\
\texttt{git diff} & \texttt{git diff id1 id2 file} to compare different version using SHA-1 ID.\\
 & \texttt{git diff HEAD\~{}3 HEAD\~{}4} : w.r.t to last commit.\\
\texttt{git commit} & To edit message in last commit.\\
\texttt{--amend} & Can also be used to modify files in previous commit, but I don't know how.\\
\hline
\end{tabularx}

\begin{tabularx}{\linewidth}{lX}
\texttt{git branch} & Creates a new branch. It also lists all branches and indicate the branch that is used currently.\\
\texttt{-d} & To delete a branch.\\
\texttt{-m} & Rename a branch. Eg: \\
& \texttt{git branch -m \textit{new-branch}} \# Renames current branch.\\
& \texttt{git branch -m \textit{old-branch} \textit{new-branch}}.\\
\texttt{--all} & To view hidden branches including remote repositories. For eg, \texttt{/remote/origin/master} is usually hidden. This functions like an actual branch but one cannot develop in this remote branch.\\
\texttt{git checkout} & To jump between branches. Use branch name that you want to jump to.\\
\texttt{git merge} & To merge two branches go to the branch you want to merge to and use \texttt{git merge <other branch>}. Merge conflict can be resolved as described earlier. In fact the earlier merge conflict was between a local branch and a remote branch.\\
\texttt{git push } & New branch from local can be synchronized with remote using: \texttt{git push origin branchname}.\\
\texttt{git fetch} & Used to synchronize my remote branch with remote remote repository. Eg: \texttt{git fetch origin}. To incoporate this to local branch use \texttt{git merge}.\\
\end{tabularx}
\underline{\textbf{NOTE:}} \texttt{git pull} is nothing but \texttt{git fetch} followed by \texttt{git merge}.\\

\texttt{git checkout -b new-methods origin/new-methods}\\ This command simultaneously creates and swithces a new branch using \texttt{-b} option. This local branch will push and pull to this specific remote branch.\\

\texttt{git remote prune origin} : To prune a stale branch in /remote branch.

\vfill \null
\columnbreak

\section{Markdown}
\textbf{Text formatting:}\\
\begin{itemize}
\item \texttt{*italics*} \\
\item \texttt{**bold**} \\
\item \texttt{*** bold italics ***}\\
\item \texttt{\_\_underline\_\_}\\
\item \texttt{\_\_*underline italics*\_\_}\\
\item \texttt{\_\_**underline bold**\_\_}
\item \texttt{\_\_***underline bold italics***\_\_}
\item \texttt{\~{}\~{}strikethrough\~{}\~{}}
\item Text coloring:\\
\texttt{$<$span style="color:blue"$>$ blue text $<$/span$>$}

\end{itemize}

\textbf{Heading, lists and links}\\

\begin{itemize}

\item Itemized list: \texttt{* item 1} or \texttt{+ item 1} or \texttt{- item 1}
\item Ordered list: Eg:\\
\texttt{1. red}\\
\texttt{2. blue} \\
\texttt{4. green} \# Here output automatically numbers it to \texttt{3} \\ 
\item Use \# for Headers. \\
\texttt{\# Header level 1}\\
\texttt{\#\# Header level 2}\\
Markdown supports upto 6 levels.\\
\item \texttt{$<$http://website.com/link$>$} \\
\item \texttt{[link text](http://website.com/link)} \\
\item Insert figure\\
\texttt{![alt text](path/to/figure.png/)}\\
\end{itemize}
\textbf{Inserting code}\\
\begin{itemize}
\item \texttt{`inline code`}, Use backticks. \\
\item Code block with tilde:\\
\texttt{\~{}\~{}\~{} Language (Optional used by pandoc to )}\\
\texttt{code block}\\
\texttt{code block}\\
\texttt{\~{}\~{}\~{}\~{}\~{}} \\
\item Codeblock with three backticks:\\
\texttt{```Language (Optional used by pandoc to )}\\
\texttt{code block}\\
\texttt{code block}\\
\texttt{```}
\end{itemize}

\section{Pandoc}

\begin{itemize}
\item \textbf{Markdown to HTML (simple version)}\\
\texttt{\$ pandoc -f markdown -t html README.md -o README.html}
\item \textbf{md to word}\\
\texttt{\$ pandoc -s README.md -o README.docx}
\item \textbf{Standalone}: \texttt{-s}. Necessary for syntax highlighting.\\
To get list of languages: \texttt{--list-highlight-languages}
\item \textbf{Box/shading for code}: Use \texttt{--highlight-style}. Eg:\\
\texttt{--highlight-style tango} \# Good for light shade.\\
\texttt{--highlight-style breezedark} \# Good for dark shade.\\
\texttt{--list-highlight-style} \# List of highlight themes.
\end{itemize}

\vfill \null
\columnbreak

\section{Uncategorized}

\textbf{Terminal shortcuts}

\begin{tabularx}{\linewidth}{lX}
\keys{ctrl} + \keys{W} & Delete from cursor to beginning of word.\\
\keys{ctrl} + \keys{U} & Delete from current cursor to start of line.\\
\keys{ctrl} + \keys{A} & Move cursor to begining of line.\\
\keys{ctrl} + \keys{E} & Move cursor to end of line.\\
\keys{ctrl} + \keys{L} & Clear the screen.\\
\keys{alt} + \keys{F} & Move forward by word.\\
\keys{alt} + \keys{B} & Move backward by word.\\
\end{tabularx}


\vfill \null
\columnbreak

\section{WSL and windows CMD}
\subsection{Execute command prompt commands from WSL.}
\begin{itemize}
	\item Notepad:\\
		\texttt{notepad.exe}\\
		\texttt{notepad.exe temp.txt}\\
	\item File explorer:\\
		\texttt{explorer.exe}\\
		\texttt{explorer.exe .}\\
	\item Execute command prompt commands in WSL.\\
		\texttt{cmd.exe \textit{commmand-line-commands}}
		Eg: Opening a windows program\\
		\texttt{cmd.exe /C start \textit{program\_name} \textit{file\_name}}\\
		Eg:\\
		\texttt{cmd.exe /C start SumatraPDF.exe mementopython3-english.pdf}
\end{itemize}

\subsection{Open from command prompt}
\begin{itemize}
	\item Websites using edge or chrome.\\
		Edge: \texttt{start microsoft-edge}\\
		Edge: \texttt{start microsoft-edge:http://www.google.co.in/}\\
	\item MS-office apps.\\
	\item Other applications.\\
\end{itemize}

\vfill \null
\columnbreak

\section{Using GUI in WSL}
\subsection{Installing XFCE}
\textcolor{red}{Under construction}\\
\textbf{Ref:}\\
https://www.youtube.com/watch?v=nKCe9UE-quA\\
https://www.shogan.co.uk/how-tos/wsl2-gui-x-server-using-vcxsrv/


\subsection{Running XFCE}
\textbf{Open XLaunch app}\\
The following is just to open a windows with simple settings.
\begin{enumerate}
\item Doulble-click and open XLaunch app. You will see a dialog box for display settings.
\item Choose "One large window" and choose "-1" for Display Number. Click "Next".
\item Choose "Start no client". Click "Next".
\item Check "Clipboard", "Primary Selection", and "Native opengl". Click "Next".
\item Save the configuration if you want, or just click "Finish" to start the window.
\end{enumerate}

\textbf{Launch xfce in WSL}\\
Execute the command \textcolor{blue}{\texttt{xfce4-session}}. Ignore the warnings.

\vfill \null
\columnbreak

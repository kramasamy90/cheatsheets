
\section{Markdown}
\textbf{Text formatting:}\\
\begin{itemize}
\item \texttt{*italics*} \\
\item \texttt{**bold**} \\
\item \texttt{*** bold italics ***}\\
\item \texttt{\_\_underline\_\_}\\
\item \texttt{\_\_*underline italics*\_\_}\\
\item \texttt{\_\_**underline bold**\_\_}
\item \texttt{\_\_***underline bold italics***\_\_}
\item \texttt{\~{}\~{}strikethrough\~{}\~{}}
\item Text coloring:\\
\texttt{$<$span style="color:blue"$>$ blue text $<$/span$>$}

\end{itemize}

\textbf{Heading, lists and links}\\

\begin{itemize}

\item Itemized list: \texttt{* item 1} or \texttt{+ item 1} or \texttt{- item 1}
\item Ordered list: Eg:\\
\texttt{1. red}\\
\texttt{2. blue} \\
\texttt{4. green} \# Here output automatically numbers it to \texttt{3} \\ 
\item Use \# for Headers. \\
\texttt{\# Header level 1}\\
\texttt{\#\# Header level 2}\\
Markdown supports upto 6 levels.\\
\item \texttt{$<$http://website.com/link$>$} \\
\item \texttt{[link text](http://website.com/link)} \\
\item Insert figure\\
\texttt{![alt text](path/to/figure.png/)}\\
\end{itemize}
\textbf{Inserting code}\\
\begin{itemize}
\item \texttt{`inline code`}, Use backticks. \\
\item Code block with tilde:\\
\texttt{\~{}\~{}\~{} Language (Optional used by pandoc to )}\\
\texttt{code block}\\
\texttt{code block}\\
\texttt{\~{}\~{}\~{}\~{}\~{}} \\
\item Codeblock with three backticks:\\
\texttt{```Language (Optional used by pandoc to )}\\
\texttt{code block}\\
\texttt{code block}\\
\texttt{```}
\end{itemize}

\section{Pandoc}

\begin{itemize}
\item \textbf{Markdown to HTML (simple version)}\\
\texttt{\$ pandoc -f markdown -t html README.md -o README.html}
\item \textbf{md to word}\\
\texttt{\$ pandoc -s README.md -o README.docx}
\item \textbf{Standalone}: \texttt{-s}. Necessary for syntax highlighting.\\
To get list of languages: \texttt{--list-highlight-languages}
\item \textbf{Box/shading for code}: Use \texttt{--highlight-style}. Eg:\\
\texttt{--highlight-style tango} \# Good for light shade.\\
\texttt{--highlight-style breezedark} \# Good for dark shade.\\
\texttt{--list-highlight-style} \# List of highlight themes.
\end{itemize}

\vfill \null
\columnbreak

\section{Uncategorized}

\textbf{Terminal shortcuts}

\begin{tabularx}{\linewidth}{lX}
\keys{ctrl} + \keys{W} & Delete from cursor to beginning of word.\\
\keys{ctrl} + \keys{U} & Delete from current cursor to start of line.\\
\keys{ctrl} + \keys{A} & Move cursor to begining of line.\\
\keys{ctrl} + \keys{E} & Move cursor to end of line.\\
\keys{ctrl} + \keys{L} & Clear the screen.\\
\keys{alt} + \keys{F} & Move forward by word.\\
\keys{alt} + \keys{B} & Move backward by word.\\
\end{tabularx}


\vfill \null
\columnbreak

\section{WSL and windows CMD}
\subsection{Execute command prompt commands from WSL.}
\begin{itemize}
	\item Notepad:\\
		\texttt{notepad.exe}\\
		\texttt{notepad.exe temp.txt}\\
	\item File explorer:\\
		\texttt{explorer.exe}\\
		\texttt{explorer.exe .}\\
	\item Execute command prompt commands in WSL.\\
		\texttt{cmd.exe \textit{commmand-line-commands}}
		Eg: Opening a windows program\\
		\texttt{cmd.exe /C start \textit{program\_name} \textit{file\_name}}\\
		Eg:\\
		\texttt{cmd.exe /C start SumatraPDF.exe mementopython3-english.pdf}
\end{itemize}

\subsection{Open from command prompt}
\begin{itemize}
	\item Websites using edge or chrome.\\
		Edge: \texttt{start microsoft-edge}\\
		Edge: \texttt{start microsoft-edge:http://www.google.co.in/}\\
	\item MS-office apps.\\
	\item Other applications.\\
\end{itemize}

\vfill \null
\columnbreak

\section{Using GUI in WSL}
\subsection{Installing XFCE}
\textcolor{red}{Under construction}\\
\textbf{Ref:}\\
https://www.youtube.com/watch?v=nKCe9UE-quA\\
https://www.shogan.co.uk/how-tos/wsl2-gui-x-server-using-vcxsrv/


\subsection{Running XFCE}
\textbf{Open XLaunch app}\\
The following is just to open a windows with simple settings.
\begin{enumerate}
\item Doulble-click and open XLaunch app. You will see a dialog box for display settings.
\item Choose "One large window" and choose "-1" for Display Number. Click "Next".
\item Choose "Start no client". Click "Next".
\item Check "Clipboard", "Primary Selection", and "Native opengl". Click "Next".
\item Save the configuration if you want, or just click "Finish" to start the window.
\end{enumerate}

\textbf{Launch xfce in WSL}\\
Execute the command \textcolor{blue}{\texttt{xfce4-session}}. Ignore the warnings.

\vfill \null
\columnbreak

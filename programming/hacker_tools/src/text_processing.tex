\section{Text processing}

\begin{tabularx}{\linewidth}{lX}

\texttt{echo} & Process and print whatever follows.\\
\texttt{echo -e} & enable backslash escapes like \texttt{\textbackslash \textbackslash, \textbackslash t, \textbackslash n} \\
\texttt{cat} & Takes standard input or input from file and gives standard output. \\
\texttt{cat -n} & Output with line numbers.\\
\texttt{head -n x} & Print first x lines.Default: 10 lines.\\
\texttt{tail -n y} & Print last y lines.\\
\texttt{wc} & Word count. Outputs number of words, lines and characters.\\
\texttt{wc -l} & Outputs only number of lines.\\
\texttt{tr} & Translate. Eg: \texttt{tr ':' '\textbackslash t'}.\\
\hline
\texttt{less} & Pager. Commonly used commands: \\
\keys{Space} & Next page.\\
\keys{b} & Previous page.\\
\keys{g} & First line.\\
\keys{G} & Last line.\\
\keys{j} & Down (One line at a time).\\
\keys{k} & Up (One line at a time).\\
\texttt{/<pattern>} & Search down for a pattern.\\
\texttt{?<pattern>} & Search up for pattern.\\
\keys{n} & Repeat last search downward.\\
\keys{N} & Repeat last search upward.\\
\hline
\texttt{cut} & To extract specific columns.\\
\texttt{-f x} & Extract columns x.\\
\texttt{-f x-z} & Extract range of columns.\\
\texttt{-f w,x-z} & Extract w and x-z. Cut cannot reorder column.\\
\texttt{-d} & Specify delimiter eg: \texttt{-d","}. Default delimiter is tab. \\
\texttt{column -t} & To visualize columns of data. Usually data is piped to \texttt{column -t}.\\
\texttt{-s}& Specify delimiter using \texttt{-s","}. Default: tab.\\
\hline

\end{tabularx}

%\vfill \null
%\columnbreak

\begin{tabularx}{\linewidth}{lX}
\texttt{grep} & Use as \texttt{grep "<pattern>" file}. Quotation around the patter is not necessary but it is safe. If the pattern contains quote then use single quotes eg: \texttt{grep '...".."..'}.\\
\texttt{-i} & Case insensitive.\\
\texttt{-E} & To use regular expressions in grep.\\
\texttt{\^} & Look for pattern in the beginning of line. Eg: \texttt{"\^{}\#"}\\
\texttt{-w} & Matches the entire word surrounded by space.\\
\texttt{-v} & Returns only lines that do not match the pattern.\\
\texttt{-o} & Return the exact matching pattern.\\
\texttt{-c} & Count how many lines match a pattern.\\
\texttt{-B1} & Print one line of context before the matching line.\\
\texttt{-A2} & Print two lines of context after the matching line.\\
\texttt{-C} & Context before and after the matching line.(\textcolor{red}{Doesn't work?})\\
\hline

\end{tabularx}

\begin{tabularx}{\linewidth}{lX}
\texttt{sort} & Sorts alphanumerically by line.\\
\texttt{-ka,b} & Sorts w.r.t to columns a to b.\\
\texttt{-k2,2n} & Treats columns 2 as numeric and sorts w.r.t to columns 2.\\
\texttt{-t} & Specify delimiter eg: \texttt{-t","}. Default = tab.\\
\texttt{-s} & Stable sort. Do not reorder lines in file if the sort rank is equal.\\
\texttt{-c} & Check if the file is already sorted. \\
\texttt{-r} & Reverse sort.\\
\texttt{-V} & Understands numbers inside string. Eg \texttt{chr22}.\\
\texttt{-S} & Specify memory to be used. \\
 & Eg: \texttt{-S 2G \# Use 2 GB}, \\ 
 & \hspace{13pt} \texttt{-S 50\% \# Use 50\% of memory}. \\
\texttt{--parallel} & to use parallel processing.\\
\hline
\texttt{uniq} & Usually used along with sort as : \texttt{sort ...| uniq}.\\
\texttt{-i} & Case insensitive.\\
\texttt{-c} & Count occurences next to the unique lines.\\
\texttt{-d} & Return line with duplicates.\\
\hline
\texttt{join} & Combine data based on a common column. Eg: \texttt{join -1 a -2 b file1 file2}. \texttt{a} and \texttt{b} represent two columns common to file1 and file2. \\
\texttt{-a} & If some elements of common column are missing from one file. Use this flag to show all elements of common column from superset file. \\
\hline

\end{tabularx}

\textcolor{red}{Checkout hexdump}

\subsection{Awk}

Format: \texttt{awk pattern \{action\} input1.txt input2.txt} \\ \texttt{awk -f file.awk input.txt}.\\ Record = row. Column = fields. \\

\begin{tabularx}{\linewidth}{lX}

\texttt{-F} & Input field separator. Eg: \texttt{awk -F"," input.txt}. Defaule field separator = tab.\\
\texttt{-f} & Take input from file. Eg: awk -f file.awk input.txt.\\
\texttt{(...) \&\& (...)} &  Use logical operators. See below.\\
\texttt{\$n ~ /.../} & Use regular expression between slashes.\\
\texttt{/.../,/.../} & Specify range. Works only with regex (with double slash) .\\
\texttt{BEGIN\{...\}} & Eg: \texttt{awk 'BEGIN\{...\} ... \{...\} END\{...\}'}\\
\texttt{END\{...\}} & \\
\hline
\end{tabularx}
\textbf{Awk operations:} \texttt{+,-,*,/,\%,\^{}}. \\
\texttt{a ... b}. Replace "\texttt{...}" : \texttt{==,!=,<,>,<=,>=,\~{},!\~{},\&\&,||,!a}\\
\textbf{Field separators:} FS,RS,OFS, ORS.\\
\textbf{Awk variables:} NF, NR (Record number accumulates between files.), FNR(Resets record number after every files.).\\

\begin{tabularx}{\linewidth}{lX}
\hline
\end{tabularx}

\textbf{Example awk script file}\\
\texttt{awk -f script.awk plasmids.tsv}\\
\texttt{BEGIN\{FS="\textbackslash t";OFS="\textbackslash t";x=0\}}\\
\texttt{/[Cc]re/\{}\\
\texttt{x+=1;}\\
\texttt{print x,\$1,\$2\}}\\
\texttt{END\{print "There are " x "plasmids with Cre"\}}

\begin{tabularx}{\linewidth}{lX}
\hline
\end{tabularx}

\textcolor{red}{Checkout BioAwk}.\\
\textcolor{red}{Checkout control flow}.\\
\begin{tabularx}{\linewidth}{lX}
\hline
\end{tabularx}

\subsection{Sed}

\texttt{sed 's/\textit{target}/\textit{replacement}/\textit{flag}'}

\begin{tabularx}{\linewidth}{lX}
\texttt{-e} & to Chain commands.Eg: \texttt{sed -e 's/:/\t/' -e 's/-/\t/'}.\\
\texttt{-E} & Use extended POSIX.\\
\texttt{g} & Global flag. Usually sed replaces only the first occurrence  in a sentence. Use global flag to replace all occurences.\\
\texttt{i} & To make the search case insensitive.\\
\end{tabularx}

%\hspace{10pt}\\
%\hspace{10pt}\\
%\hspace{10pt}\\

%\pagebreak

\subsection{Regular Expression}

\textbf{Single character meta characters}
\begin{tabularx}{\linewidth}{lX}

\texttt{.} & Match any single character.\\
\texttt{[ ]} & Match any single character between \texttt{[]}. Eg: \texttt{[at]} match "a" or "t".\\
\texttt{[\^{}]} & Match any single charcter except on between \texttt{[]}.\\
\texttt{[0-9]} & Any number between 0 and 9. Eg: \texttt{0-3a-cz]} equals \texttt{[123abcz]}.\\
\texttt{(...)} & Grouping. eg: \texttt{(AT)+} or \texttt{(GLY ) \{2,\}}.\\
\hline

\end{tabularx}


\textbf{Quantifiers}
\begin{tabularx}{\linewidth}{lX}
\texttt{?} & Match preceding character zero or one time. \\
\texttt{*} & Match zero or more time.\\
\texttt{+} & Match one or more time.\\
\texttt{\{n\}} & Match n times.\\
\texttt{\{n,\}} & Match atleast n times.\\
\texttt{\{a,b\}} & Match atleast a times, atmost b times.\\

\hline

\end{tabularx}

\textbf{Anchors}

\begin{tabularx}{\linewidth}{lX}

\texttt{\^{}} & Match the start of a line.\\
\texttt{\$} & Match end of a line.\\
\texttt{\textbackslash $<$} & Match beginning of word.\\
\texttt{\textbackslash $>$} & Match at the end of word.\\
\texttt{\textbackslash b} & Match either beginning or end of word.\\
\texttt{\textbackslash B} & Match any character not at the beginning or end. \\

\hline

\end{tabularx}

\textbf{Character class}

[:alnum:], [:digit:], [:alpha:], [:upper:], [:lower:], [:blank:], [:space:], [:punct:] and [:print:].\\

Use backslash as escape character. \\
\begin{tabularx}{\linewidth}{lX}
\texttt{\textbackslash s} & white space character. What it includes depends on the flavour of regex.\\
\texttt{\textbackslash d} & Add digits.\\
\texttt{\textbackslash w} & Word character, matches \texttt{[A-Za-z0-9\_]}\\
\end{tabularx}
\texttt{|} as OR logical operator: \texttt{(GLY|GLN)}.
\texttt{"one and|or two"} is equal to \texttt{"(one and)|(or two)"}.\\
\texttt{"one (and|or) two"} is "one and two" or "one or two".\\

\textbf{Back references:} () : Memorizes the match for regular expression within parenthesis. Use \texttt{\textbackslash n} to recall nth match.

\begin{tabularx}{\linewidth}{lX}
\hline
\end{tabularx}

\vfill\null
\columnbreak


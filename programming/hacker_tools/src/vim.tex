\section{Vim}

\textbf{Motion}
Usage: \texttt{<num> <motion>}
\begin{tabularx}{\linewidth}{lX}
\texttt{h l} & One character left or right.\\
\texttt{j k} & One line up or down.\\
\texttt{w b} & One word forward or backwarks.\\
\texttt{e} & Simillar to w but keeps the cursor at the end of the word.\\
\texttt{0} & Cursor to the begining of the sentence.\\
\texttt{\$} & Moves cursor to the end of the sentences.\\
\texttt{G} & End of the file.\\
\texttt{gg} & First line.\\
\texttt{H} & Top of screen.\\
\texttt{M} & Middle of screen.\\
\texttt{L} & Botom of screen.\\
\texttt{<num>G} & Go to line \texttt{<num>}.\\
\keys{\texttt{Ctrl}} + \texttt{f} & One screen forward.\\
\keys{\texttt{Ctrl}} + \texttt{b} & One screen backward.\\
\keys{\texttt{Ctrl}} + \texttt{G} & View position in the file.\\
\keys{\texttt{Ctrl}} + \texttt{O} & Go to where you came from .\\
\keys{\texttt{Ctrl}} + \texttt{I} & Opposite of \keys{\texttt{Ctrl}} + \keys{\texttt{O}}\\
\texttt{\%} & Go to the corresponding opening or closing parenthesis.\\
\hline\\

\end{tabularx}

\textbf{Operators}

\begin{tabularx}{\linewidth}{lX}
\texttt{i} & INSERT mode\\
\texttt{a} & append, goes to insert mode\\
\texttt{a} & append from the end of the line.\\
\texttt{v} & visual selection, selection is stored in clipboard\\
\texttt{o} & open a line below\\
\texttt{O} & open a line above\\
\keys{\texttt{Esc}} & Go to command mode\\
\texttt{d} & delete and also cut, $\equiv$ \keys{\texttt{Ctrl}} + \keys{\texttt{X}}\\
\texttt{dd} & delete whole sentence\\
\texttt{x} & delete character under the cursor\\
\texttt{r} & replace the character under the cursor\\
\texttt{R} & replace until \keys{\texttt{Esc}} \\
\texttt{c} & change: works equivalent to \texttt{d} followed by \texttt{i}\\
\texttt{y} & yank, copy\\
\texttt{p} & paste\\
\texttt{u} & undo most recent edit\\
\texttt{U} & undo all the changes in the line\\
\keys{\texttt{Ctrl}} + \keys{\texttt{R}} & Redo\\
\hline

\end{tabularx}

\textbf{Copy, paste, bookmark}

\begin{tabularx}{\linewidth}{lX}

\texttt{:xmy} & Move line \texttt{x} below line \texttt{y}.\\
\texttt{:x,ymz} & Moves lines between and including \texttt{x} and \texttt{y} below line \texttt{z}.\\
\texttt{:xty} & Copy line \texttt{x} below line \texttt{y}.\\
\texttt{:x,ytz} & Copy lines between and including \texttt{x} and \texttt{y} below line \texttt{z}.\\
\texttt{ma} & Set bookmark at current line. \texttt{a} $\in$ [a-z].\\
\texttt{'a} & Jump to bookmark \texttt{a}.\\
\texttt{:'a,'bco'c} & Copy lines between and including bookmarks \texttt{a} and \texttt{b} below bookmark c.\\
\texttt{:'a,'bco'z} & Copy lines between and including bookmarks \texttt{a} and \texttt{b} below line z.\\
\hline\\
\end{tabularx}

\textbf{Search and replace}\\

\begin{tabularx}{\linewidth}{lX}
:/REGEX & Find regular expression.\\
n & next search target\\
N & Previous search target\\
:s/target/replace & Simillar to sed. Replaces target only in the current sentence and only once.\\
:s/target/replace/g & Replaces at all instance in the current sentence.\\
:\%s/target/replace/g & Replaces through the entire file.\\
:\%s/target/replace/gc & Ask for confirmation at each instance.\\

\end{tabularx}

\begin{tabularx}{\linewidth}{lX}
\hline
\end{tabularx}

\textbf{Save, write and Exit}

\begin{tabularx}{\linewidth}{lX}
\texttt{:q} & quit\\
\texttt{:q!} & quit without saving\\
\texttt{:w} & save the current file\\
\texttt{:wq} or \texttt{:x} & save and quite \\
\texttt{:w \textit{file}} & write to \textit{file}.\\
\texttt{:xyw \textit{file}} & write lines between and including lines \texttt{x} and \texttt{y} to \textit{file}.\\
\texttt{:!} & Execute shell command. Eg: \texttt{:!pwd}\\
\hline\\

\end{tabularx}

\textbf{\texttt{:set}}\\
Usage: \texttt{:set \textit{option}}. Eg: \texttt{:set ic}
\begin{tabularx}{\linewidth}{lX}
ic & Case-insensititve search\\
hls & Highlight search\\
number & Show line number \\
\end{tabularx}
To turnoff the option use no. Eg \texttt{:set noic} to turnoff \texttt{ic}\\
\begin{tabularx}{\linewidth}{lX}
\hline\
\end{tabularx}
\textbf{Etc}

\keys{Ctrl} + \keys{D} for command completion.\\
\keys{Tab} for filename completion.\\
For further setting: \texttt{$\sim$/.vimrc} \\
\textbf{Help}:\\
\keys{F1}\\
\texttt{:help}\\
\texttt{:help w}\\
\texttt{:help user-manual}\\

\textbf{Default settings:}
Set default settings in \texttt{~/.vimrc} \\
Create this file if it does not exist. \\
Example \texttt{.vimrc} file:\\

\begin{mdframed}
\texttt{syntax on}\\
\texttt{colorscheme desert}\\
\texttt{set number}\\
\texttt{set hls}

\end{mdframed}

\vfill\null

\columnbreak


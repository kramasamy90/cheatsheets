\documentclass[10pt,landscape]{article}
\usepackage[utf8]{inputenc}
\usepackage[left=0.25in,right=0.25in,top=0.25in,bottom=0.5in,a4paper, landscape]{geometry}
\setlength{\parindent}{0pt}
\setlength{\parskip}{0pt plus 0.5ex}
\usepackage[x11names]{xcolor} 
% \linespread{1.15} % Line spacing.


% Title sec:
% ==========

\usepackage{titlesec} % !Incompatible with beamer.
\titlelabel{\thetitle.\quad} % Dot after counter
\titleformat*{\section}{\Large\bfseries\color{Firebrick4}}
\titleformat*{\subsection}{\large\bfseries\color{Blue4}}
\titleformat*{\subsubsection}{\normalsize\bfseries\center}
\titlespacing*{\section}{0mm}{1ex plus 0.2ex minus 0.2ex}{0.5ex plus .1ex}
\titlespacing*{\subsection}{0mm}{1ex plus 0.2ex minus 0.2ex}{0.5ex plus .1ex}
\titlespacing*{\subsubsection}{0mm}{1ex plus 0.2ex minus 0.2ex}{0.5ex plus .1ex}
\setcounter{secnumdepth}{2}


% Mathematics:
% ============

\usepackage{amssymb,amsmath,amsthm,amsfonts}
\usepackage{centernot}
\usepackage{mathtools}
\usepackage{mathrsfs}

\everymath{\displaystyle} % Keeps same font size for inline math. For eg. in fractions.


\newtheoremstyle{mythmstl}{}{}{\itshape}{}{\bfseries}{:}{\newline}{\thmname{#1}\thmnumber{ #2}\thmnote{ (#3)}}%
\theoremstyle{mythmstl}
\newtheorem{theorem}{Theorem}[section]
\theoremstyle{plain} % default
\newtheorem{corollary}{Corollary}[theorem]
\newtheorem{lemma}[theorem]{Lemma}
\newtheorem{proposition}{Proposition}

\newtheoremstyle{mydefstl}{}{}{}{}{\bfseries}{:}{\newline}{\thmname{#1}\thmnumber{ #2}\thmnote{ (#3)}}%
\theoremstyle{mydefstl}
\newtheorem{definition}{Definition}[section]
\newtheorem{example}{Example}[section]
\newtheorem{exercise}{Exercise}[section]

\theoremstyle{remark}
\newtheorem*{remark}{Remark}
\newtheorem*{note}{Note}
\newtheorem*{case}{Case}


% New declarations
% ================
\usepackage{mathtools}
\DeclarePairedDelimiter\ceil{\lceil}{\rceil}
\DeclarePairedDelimiter\floor{\lfloor}{\rfloor}
\DeclarePairedDelimiter\angbrac{\langle}{\rangle}


% Tables and Box:
%================
\usepackage{tabularx} % To use table that wrap text and fit the column width.
\usepackage{multicol,multirow}
\usepackage{booktabs} % Necessary for \toprule \midrule in tables.
\usepackage{tabu} % Better tables.
\usepackage[framemethod=TikZ]{mdframed} % To make text boxes.


% URL
%========
\usepackage{url} % For \url{} Alternate for above.
\usepackage[colorlinks]{hyperref}
% USAGE: \href{url}{alias}
\hypersetup{citecolor=pink}
\hypersetup{linkcolor=red}
\hypersetup{urlcolor=blue}


% Others:
%========
\usepackage{enumitem}
\usepackage[os=win]{menukeys} % To use text in box that looks like keyboard keys.

\begin{document}

\raggedright
\footnotesize
\begin{multicols*}{3}
\setlength{\premulticols}{1pt}
\setlength{\postmulticols}{1pt}
\setlength{\multicolsep}{1pt}
\setlength{\columnsep}{2pt}

%%%%%%%%%%%%%%%%%%%%%%%%%%%%%%%%%%%%%%%%%%%%%%%%%%%%%%%%%%%%%%%%%%%%%%%%%%%%%%%%%%%%%%%%%%%%%%%%%%
\begin{center}
\begin{tabular}{c}
\LARGE{\textbf{CUDA}}\\
\today\\
Ramasamy Kandasamy\\
\end{tabular}
\end{center}

%%%%%%%%%%%%%%%%%%%%%%%%%%%%%%%%%%%%%%%%%%%%%%%%%%%%%%%%%%%%%%%%%%%%%%%%%%%%%%%%%%%%%%%%%%%%%%%%%%

\section*{DeviceManagement}

\begin{itemize}
	\item \texttt{cudaDeviceSynchronize()}: \\
    Tells CPU to wait for GPU code execution.
    \item \texttt{cudaGetDeviceProperty(cudaDeviceProp* prop, int Device)}\\
    \texttt{Device}: Device number, eg: \texttt{0}. Store device property in the structure pointed by \texttt{prop}. 
\end{itemize}

\section*{Memory Management}

\begin{itemize}
	\item \texttt{cudaMalloc(void** (\&dp), size\_t sz)}:\\
    Allocates a memory of size \texttt{sz} and store the pointer value at \texttt{dp}, where dp is a pointer.
    \item \texttt{cudaMemset(void* da, char v, size\_t sz)}:\\
    This function copies the byte-sized value \texttt{v}, starting from the memory location \texttt{da} \textbf{byte-by-byte} upto \texttt{sz} bytes.\\
    textcolor{red}{IMPORTANT:} The following will give unexpected results:\\
    \textcolor{red}{\texttt{cudaMemset(da, 0, sizeof(int))}}
    \item \texttt{cudaMemcpy(void* da, void* ha, size\_t sz, cudaMemcpyHostToDevice)}\\
    \item \texttt{cudaMemcpy(void* ha, void* da, size\_t sz, cudaMemcpyDeviceToHost)}\\
\end{itemize}

\section*{Kernels}

\subsubsection*{Types of functions}

\begin{itemize}
	\item \texttt{\_\_global\_\_}: Can be called from CPU/GPU(?), runs on GPU.
	\item \texttt{\_\_device\_\_}: Can be called only from GPU, runs on GPU. 
	\item \texttt{\_\_host\_\_}: Can be called only from CPU, runs on CPU.
\end{itemize}

If a function is decorated by both \texttt{\_\_host\_\_} and \texttt{\_\_device\_\_}, then two instances of the function are created, one for the device and another for host.\\
However if a function in device is to be called by host it should be decorated by \texttt{\_\_global\_\_}.\\
\textcolor{red}{NOTE:} Functions decorated by \texttt{\_\_global\_\_} should only return \texttt{void}.

\subsubsection*{Grids and Blocks}

\begin{mdframed}[backgroundcolor=gray!10,linecolor=Firebrick4]
\begin{verbatim}
dkernel<<<1, 1>>>(); // One block with one thread. 
dkernel<<<1, M>>>(); // One block with M threads. 
dkernel<<<N, 1>>>(); // N blocks with one thread each.
dkernel<<<N, M>>>(); // N blocks with M threads each.
\end{verbatim}
\end{mdframed}

\textbf{Grids} contain \textbf{blocks} which contain \textbf{threads.}.\\
\textbf{}

\subsubsection*{Advanced}

\begin{mdframed}[backgroundcolor=gray!10,linecolor=Firebrick4]
	\begin{verbatim}
		dim3 block(x, y); // Contains x * y threads.
		dim3 block(x, y, z); // Contains x * y * z threads.
		dim3 grid(a, b);  // Contains x * y blocks.
		dim3 grid(a, b, c);  // Contains x * y * z blocks.
		dkernel<<<grid, block>>>(da);
	\end{verbatim}
	Here, the above kernel will run a * b * c * x * y * z threads.
\end{mdframed}
\textbf{Attributes}\\
\begin{itemize}
	\item \texttt{threadIdx.x}, \texttt{threadIdx.y} and \texttt{threadIdx.z}
	\item \texttt{blockIdx.x}, \texttt{blockIdx.y} and \texttt{blockIdx.z}
	\item \texttt{blockDim.x}, \texttt{blockDim.y} and \texttt{blockDim.z}.
	\item \texttt{gridDim.x}, \texttt{gridDim.y} and \texttt{gridDim.z}.
\end{itemize}

\vfill\null
\columnbreak

 


\vfill\null
\columnbreak

\section*{Errors}
\subsection*{Detecting error in CUDA}

\textbf{Ref:}\\
https://stackoverflow.com/a/14038590/5607735

\begin{verbatim}
#define gpuErrchk(ans){gpuAssert((ans), __FILE__, __LINE__);}
inline void gpuAssert(cudaError_t code, char *file,
            int line, bool abort=true)
{
   if (code != cudaSuccess)
   {
      fprintf(stderr,"GPUassert: %s %s %d\n",
       cudaGetErrorString(code), file, line);
      if (abort) exit(code);
   }
} 
\end{verbatim}

\subsection*{Notable errors in CUDA}
\subsubsection*{1. Incomplete output}

\begin{verbatim}
__global__ void per_row_AB_kernel(long int m, long int n){
    long int K = blockIdx.x * 1024 + threadIdx.x;
    if(K >= (m * n)) return;
    printf("%ld \n", K);	
}

int main(){
    long int m, n;
    scanf("%ld", &m);
    scanf("%ld", &n);
    
    dim3 grid_1((m * n)/1024 + 1, 1, 1);
    dim3 block_1(1024, 1, 1);
    
    per_row_AB_kernel<<<grid_1, block_1>>>(m, n);
    cudaDeviceSynchronize();                                                           
}	
\end{verbatim}

In the above code for large values of \texttt{m} and \texttt{n} not all values of \texttt{K} will be printed. This is because there is limited space for in output buffer. It is intended for small scale debug style output not large scale output.\\
\textbf{Source:} https://stackoverflow.com/a/15421935/5607735
\end{multicols*}
\end{document}
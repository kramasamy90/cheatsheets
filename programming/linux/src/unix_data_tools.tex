\section{Text processing}

\begin{tabularx}{\linewidth}{lX}

\texttt{echo} & Process and print whatever follows.\\
\texttt{echo -e} & enable backslash escapes like \texttt{\textbackslash \textbackslash, \textbackslash t, \textbackslash n} \\
\texttt{cat} & Takes standard input or input from file and gives standard output. \\
\texttt{cat -n} & Output with line numbers.\\
\texttt{>, >>} & Write and append, respectively, standard output to a file. \\
\texttt{2>, 2>>} & Write and append standard error to a file. \\
\texttt{2>\&1} & Redirects std.err to std.out.\\
\texttt{<} & Take input.\\
\texttt{|} & Pipe \\
\texttt{tee} & Eg: \texttt{prog1 in.txt | tee intermediate.txt | prog > result.txt}\\
\texttt{head -n x} & Print first x lines.Default: 10 lines.\\
\texttt{tail -n y} & Print last y lines.\\
\texttt{wc} & Word count. Outputs number of words, lines and characters.\\
\texttt{wc -l} & Outputs only number of lines.\\
\texttt{tr} & Translate. Eg: \texttt{tr ':' '\textbackslash t'}.\\
\hline
\texttt{less} & Pager. Commonly used commands: \\
\keys{Space} & Next page.\\
\keys{b} & Previous page.\\
\keys{g} & First line.\\
\keys{G} & Last line.\\
\keys{j} & Down (One line at a time).\\
\keys{k} & Up (One line at a time).\\
\texttt{/<pattern>} & Search down for a pattern.\\
\texttt{?<pattern>} & Search up for pattern.\\
\keys{n} & Repeat last search downward.\\
\keys{N} & Repeat last search upward.\\
\hline
\texttt{cut} & To extract specific columns.\\
\texttt{-f x} & Extract columns x.\\
\texttt{-f x-z} & Extract range of columns.\\
\texttt{-f w,x-z} & Extract w and x-z. Cut cannot reorder column.\\
\texttt{-d} & Specify delimiter eg: \texttt{-d","}. Default delimiter is tab. \\
\texttt{column -t} & To visualize columns of data. Usually data is piped to \texttt{column -t}.\\
\texttt{-s}& Specify delimiter using \texttt{-s","}. Default: tab.\\
\hline

\end{tabularx}

%\vfill \null
%\columnbreak

\begin{tabularx}{\linewidth}{lX}
\texttt{grep} & Use as \texttt{grep "<pattern>" file}. Quotation around the patter is not necessary but it is safe. If the pattern contains quote then use single quotes eg: \texttt{grep '...".."..'}.\\
\texttt{-i} & Case insensitive.\\
\texttt{-E} & To use regular expressions in grep.\\
\texttt{\^} & Look for pattern in the beginning of line. Eg: \texttt{"\^{}\#"}\\
\texttt{-w} & Matches the entire word surrounded by space.\\
\texttt{-v} & Returns only lines that do not match the pattern.\\
\texttt{-o} & Return the exact matching pattern.\\
\texttt{-c} & Count how many lines match a pattern.\\
\texttt{-B1} & Print one line of context before the matching line.\\
\texttt{-A2} & Print two lines of context after the matching line.\\
\texttt{-C} & Context before and after the matching line.(\textcolor{red}{Doesn't work?})\\
\hline

\end{tabularx}

\begin{tabularx}{\linewidth}{lX}
\texttt{sort} & Sorts alphanumerically by line.\\
\texttt{-ka,b} & Sorts w.r.t to columns a to b.\\
\texttt{-k2,2n} & Treats columns 2 as numeric and sorts w.r.t to columns 2.\\
\texttt{-t} & Specify delimiter eg: \texttt{-t","}. Default = tab.\\
\texttt{-s} & Stable sort. Do not reorder lines in file if the sort rank is equal.\\
\texttt{-c} & Check if the file is already sorted. \\
\texttt{-r} & Reverse sort.\\
\texttt{-V} & Understands numbers inside string. Eg \texttt{chr22}.\\
\texttt{-S} & Specify memory to be used. \\
 & Eg: \texttt{-S 2G \# Use 2 GB}, \\ 
 & \hspace{13pt} \texttt{-S 50\% \# Use 50\% of memory}. \\
\texttt{--parallel} & to use parallel processing.\\
\hline
\texttt{uniq} & Usually used along with sort as : \texttt{sort ...| uniq}.\\
\texttt{-i} & Case insensitive.\\
\texttt{-c} & Count occurences next to the unique lines.\\
\texttt{-d} & Return line with duplicates.\\
\hline
\texttt{join} & Combine data based on a common column. Eg: \texttt{join -1 a -2 b file1 file2}. \texttt{a} and \texttt{b} represent two columns common to file1 and file2. \\
\texttt{-a} & If some elements of common column are missing from one file. Use this flag to show all elements of common column from superset file. \\
\hline

\end{tabularx}

\textcolor{red}{Checkout hexdump}

\section{Useful Libraries}

\subsubsection{Serialization}

\begin{center}
    \large{\texttt{pickle}}\\
\end{center}
 
\texttt{import pickle as pkl}
\begin{itemize}
\item Save an object:\\
        \texttt{pkl.dump(foo, open('foo.pkl', 'wb'))}.
\item Load and object:\\
        \texttt{foo = pkl.load(open('foo.pkl', 'rb'))}.
\end{itemize}


\begin{center}
    \large{\texttt{yaml}}\\
\end{center}

Mostly used to store configuration files.\\

\texttt{import yaml}

\begin{itemize}
\item Read an YAML file.\\ 
        YAML files are loaded to a dictionary object.\\
        \texttt{with open('config.yaml', 'r') as file:}\\
        \hspace{8pt}\texttt{config = yaml.safe\_load(file)}\\
\item Save an YAML file:\\
        \texttt{with open('config.yaml', 'r') as file:}\\
        \hspace{8pt}\texttt{yaml.dump(yaml, file, default\_flow\_style=False)}
        NOTE: Setting \texttt{default\_flow\_style} to \texttt{False} writes the output in block style and has better readability.\\
\end{itemize}

\textbf{Examples of YAML files}

\begin{mdframed}[backgroundcolor=gray!10,linecolor=Firebrick4]
\begin{verbatim}
# Comments
n: 10 # Integer.
x: 0.1 # Float.
s: "Hello, World!" # String.
list_1: [1, 2, 3] # 1D List.
list_2: [1, 2, 3
          4, 5, 6] # A list can span multiple lines.
mat:
    - [1, 2, 3]
    - [4, 5, 6]
    - [7, 8, 9] # A list of list.

inventory: # List of dictionaries.
    - id: 123
      desc: apple
    - id: 124
      desc: mango

description: |
  This is a multiline string.
  Newlines and indentation
  will be preserved.
  
  It can contain empty lines,
  special characters, etc.

\end{verbatim}
\end{mdframed}

\columnbreak


\begin{mdframed}[backgroundcolor=gray!10,linecolor=Firebrick4]
\begin{verbatim}

description: >
  This is a multiline string
  that will be folded into a
  single line when loaded.
  
  Empty lines will be preserved
  as line breaks.
\end{verbatim}
\end{mdframed}



\subsection{Copy}

\texttt{import copy}

\begin{itemize}
\item \texttt{copy.copy()}. Creates a shallow copy.\\
        Creates a copy of the original object, 
        but not of the objects contained within this original object.\\
\item \texttt{copy.deepcopy()}. Creates new copy of the original object
        and all the objects within this original object.\\
\end{itemize}



\subsection{Datetime}

\begin{center}
   \large{\texttt{datetime.date}} 
\end{center}
Represents date.
\begin{itemize}
\item \texttt{datetime.date(Y, M, D)}.\\
        Eg: \texttt{d = datetime.date(1903, 3, 14)}.\\
        Individual components:
        \begin{itemize}
        \item \texttt{year = d.year}
        \item \texttt{month = d.month}
        \item \texttt{day = d.day}
        \end{itemize}
\item \texttt{datetime.today()}.\\
\end{itemize}

\begin{center}
    \large{\texttt{datetime.time}}
\end{center}
Represents time.

\begin{itemize}
\item \texttt{t = datetime.time(H, m, s)}.\\
    Eg: \texttt{t = datetime.time(23, 15, 17)}.\\
    \begin{itemize}
    \item \texttt{hr = t.hour}\\
    \item \texttt{min = t.minute}\\
    \item \texttt{sec = t.second}\\
    \end{itemize}
\end{itemize}

\begin{center}
    \large{\texttt{datetime.datetime}}
\end{center}

Represents both date and time.

\begin{itemize}
    \item \texttt{today = datetime.datetime.today()}
    \item \texttt{now = datetime.datetime.now()}.
    \item \texttt{dt = datetime.combine(d, t)}.
\end{itemize}


\begin{center}
    \large{\texttt{datetime.timedelta}}
\end{center}

\begin{itemize}
\item \texttt{delta = datetime.timedelta(hours = 4)}
\item \texttt{t2 = t1 + delta}
\item \texttt{delta = t2 - t2}. \texttt{delta} is an instance of \texttt{timedelta}.\\
\end{itemize}

\subsubsection{Etc.}

\begin{itemize}
\item \texttt{d.strftime("\%Y\%m\%d\%H\%M\%S")}
\item \texttt{d = datetime.strptime(date\_string, "\%Y\%m\%d\%H\%M\%S")}
\item \texttt{d.replace(year = 2025)}
\item \texttt{d.weekday()}\\
        0 $\to$ Monday, 6 $\to$ Sunday.\\
\end{itemize}  

\columnbreak

\textbf{Formats:}\\
\vspace{2pt}
\begin{tabular}{lcl|lcl}
    \hline
    \texttt{\%Y}  & - & 2023     &   \texttt{\%H}  & - &    24H\\
    \texttt{\%y}  & - & 23       &   \texttt{\%I}  & - &    12H\\
    \texttt{\%m}  & - & 01       &   \texttt{\%M}  & - &    mins\\
    \texttt{\%-m} & - & 1        &   \texttt{\%S}  & - &    secs\\
    \texttt{\%d}  & - & 01       &   \texttt{\%f}  & - &    $\mu$s\\
    \texttt{\%-d} & - & 1        &   \texttt{\%p}  & - &    AM/PM\\
    \hline
\end{tabular} 


\begin{tabularx}{\linewidth}{lcX}
    
\texttt{\%U}    & $\to$ & First sunday of the year is start of week 01.\\
                &       & Days before that belong to week 00.\\ 
\texttt{\%W}    & $\to$ & First monday of the year is start of week 01.\\
                &       & Days before that belong to week 00.\\ 
\texttt{\%V}    & $\to$ & Week number by ISO system.\\

\end{tabularx}

\subsection*{Regular Expression}

\texttt{import re}\\
\textbf{Usage for finding patterns:} \texttt{re.<method>(pattern, text)}\\

\textbf{Methods in re.}\\
\begin{tabularx}{\linewidth}{lX}
    \texttt{search}         & Return the first match as a match object.\\
    \texttt{match}          & Match only at the beginning of the string. Returns a match object.\\
    \texttt{findall}        & Returns a list of all non-overlapping matches.\\
    \texttt{finditer}       & Returns an iterator to non-overlapping matches.\\
    \texttt{sub}            & Replace pattern. Usage: \texttt{re.sub(pattern, replacement, text)}.\\
    \texttt{split}          & Split by pattern. Eg: Split by white spaces.\\
                            & \texttt{pattern = r"\textbackslash s+"}\\
                            & \texttt{re.split(pattern, text)}\\
    \texttt{group}          & Returns the entire match as string.\\
    \texttt{group(n)}       & Returns the n-th group, n=0 is same as \texttt{.group()}\\
    \texttt{groups}         & Returns a tuple of all the matching groups.\\
\end{tabularx}

\textbf{Patterns}

\begin{tabularx}{\linewidth}{lX}
    \texttt{[]}             & Match characters inside.\\
    \texttt{.}              & Match any char except newline.\\
    \texttt{$\wedge$}       & Match beginning of line.\\
                            & Inside [], it negates the pattern.\\
    \texttt{\$}             & Match any char except newline.\\
    \texttt{d}              & Digits.\\
    \texttt{D}              & Any character other than digits.\\
    \texttt{w}              & Any alphanumeric.\\
    \texttt{W}              & Any character other than alphanumeric.\\
    \texttt{s}              & Any white space character: space, tab, linebreak.\\
    \texttt{S}              & any non-whitespace character.\\
    \texttt{*}              & 0 or more repitition.\\
    \texttt{+}              & 1 or more repitition.\\
    \texttt{?}              & 0 or 1 repitition.\\
    \{n\}                   & Exactly n repititions.\\
    \{m, n\}                & Between m and n repitions.\\
    ()                      & Group patterns.\\
\end{tabularx}

\begin{mdframed}[backgroundcolor=gray!10,linecolor=Firebrick4]
\begin{verbatim}
pattern = r"(\d{4})-(\d{2})-(\d{2})"
text = "2024-08-24"

match = re.match(pattern, text)

if match:
    print(match.group())  # Output: '2024-08-24'. 
    print(match.group(1)) # Output: '2024' (first group).
    print(match.group(2)) # Output: '08' (second group).
    print(match.group(3)) # Output: '24' (third group).
    print(match.groups()) # Output: ('2024', '08', '24').
\end{verbatim}
\end{mdframed}

\pagebreak
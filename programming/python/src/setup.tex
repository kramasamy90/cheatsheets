\section{Project setup, etc.}

\subsection{Virtual environments and package management}

\subsubsection{Virtual Environments}

\begin{mdframed}[backgroundcolor=gray!10,linecolor=Firebrick4]
\begin{verbatim}
python3 -m venv myenv # Create an environment.
source myenv/bin/activate # Activate the environment.
deactivate # Deactivate the environment.
\end{verbatim}
\end{mdframed}


\subsubsection*{Pip}

\begin{tabularx}{\linewidth}{lX}
    install     & Eg: \texttt{pip3 install pkg}.\\
                & \texttt{pip3 install pkg==<ver-num>}. Install specific version.\\
                & \texttt{pip3 install --upgrade pkg}.\\
                & \texttt{pip3 install -r requirements.txt}.\\
    list        & \texttt{pip3 list}.\\
    uninstall   & \texttt{pip3 uninstall pkg}.\\
    freeze      & \texttt{pip3 freeze > requirements.txt}\\
\end{tabularx}


\subsection{Documentation}

Documentation using \texttt{sphinx}.\\

\begin{itemize}
\item Install: \texttt{pip3 install sphinx}.\\
\item Initialize a docs directory: \texttt{sphinx-quickstart}.\\
        This will create \texttt{source} and \texttt{build} directories.\\
\item \texttt{source/conf.py}: Setup configurations.\\
\item \texttt{source/index.rst}: Starting point for the documentation.\\
\item \texttt{source/modules.rst}: Add modules here.\\
\item \texttt{make html}: Execute from the directory containing the makefile.\\
\item \texttt{make clean}: To clean the existing build.\\
\end{itemize}

\textbf{conf.py}\\
The template is generated by \texttt{sphinx-quickstart}\\
\begin{mdframed}[backgroundcolor=gray!10,linecolor=Firebrick4]
\begin{verbatim}
<---Template by sphinx --->
# Add path.
import sys
sys.path.append(path/to/src)

# Configure extension.
extensions = [
    'sphinx.ext.autodoc'
    'sphinx.ext.autosummary'
    'sphinx.ext.mathjax'
    'sphinx.ext.napoleon'
    'sphinx.ext.viewcode'
]

# Configure theme.
# Default is alabaster.
# Install furo with pip.
html_theme = 'furo' 
\end{verbatim}
\end{mdframed}

\columnbreak

\textbf{Example: \texttt{modules.rst}}
\begin{mdframed}[backgroundcolor=gray!10,linecolor=Firebrick4]
\begin{verbatim}
src
===

.. toctree::
   :maxdepth:4

   myclass
\end{verbatim}
\end{mdframed}



\subsection{Testing}

Unit test using \texttt{unittest} module.

\textbf{Test directory}:\\
\begin{mdframed}[backgroundcolor=gray!10,linecolor=Firebrick4]
\begin{verbatim}
myrepo/
|-- test/
    |-- __init.py__    # Could be empty but necessary.
    |-- MyClassTest.py # Tests for MyClass.py.
\end{verbatim}
\end{mdframed}


\textbf{Simple test file:}\\

\begin{mdframed}[backgroundcolor=gray!10,linecolor=Firebrick4]
\begin{verbatim}
import unittest

class MyClassTest(unittest.TestCase):
    def setup(self):
    # Setup vars etc.

    def test_foo(self):
    # Write tests.
    self.assertAlmostEqual(obs, exp)

if __name__ == '__main__':
    unittest.main()

\end{verbatim}
\end{mdframed}

\textbf{Using unittest:}
\begin{itemize}
    \item \texttt{\$ python3 -m unittest discover}: Run all the tests.
    \item \texttt{\$ python3 -m unittest myclasstest.py}: Run only \texttt{myclasstest.py}. 
    \item \texttt{\$coverage run -m unittest discover}: Run the test with coverge to get coverage report.
    \item \texttt{\$ coverage report}: Prints coverage report to stdout.
    \item \texttt{\$ coverage html}: Generates an html report.\\
            This is stored in \texttt{htmlcov} directory.\\
\end{itemize}



\pagebreak